
\documentclass{beamer}
\usepackage[utf8]{inputenc}
\usepackage[brazilian]{babel}
\usepackage{graphicx}
\inputencoding{latin1}
\usetheme{metropolis}           % Use metropolis theme
\title{Sistema de Acesso \`a Cole\c{c}\~ao de Bact\'erias Aer\'{o}bias Formadoras de End\'{o}sporos}
\date{\today}
\author{J\^{o}natas Ribeiro Senna Pires}
\institute{Univesidade de Bras\'{i}lia}
\begin{document}
  \maketitle
  \section{Objetivos}
  \begin{frame}{Objetivos}
    \begin{itemize}
        \item Criar um banco de dados para armazenar os dados coletados pelo LaBafes;
        \item Criar uma plataforma de acesso \`as informa\c{c}\~oes contidas no banco de dados;
        \item Permitir que os membros do LaBafes possam ter controle do banco de dados:\\
        \begin{enumerate}
                \item Adi\c{c}\~ao de linhagens;
                \item Edi\c{c}\~ao de linhagens;
                \item Remo\c{c}\~ao de linhagens;
                \item Controle sobre as altera\c{c}\~oes realizadas.
        \end{enumerate}
    \end{itemize}
  \end{frame}
  
  \section{Banco de Dados}
  \begin{frame}{Diagrama Entidade Relacionamento}
    \includegraphics[height=200pt, width=325pt]{CBafesDER1}
    \\ * Entidades com o mesmo nome representam a mesma entidade
  \end{frame}
  \begin{frame}{Diagrama Entidade Relacionamento}
    \includegraphics[height=200pt, width=325pt]{CBafesDER2}
    \\ * Entidades com o mesmo nome representam a mesma entidade
  \end{frame}
  \begin{frame}{Sistema Gerenciador de Banco de Dados}
    \begin{itemize}
        \item SGBD escolhido PostgreSQL
        \begin{itemize}
            \item OpenSource;
            \item Bem documentado.
        \end{itemize}
    \end{itemize}
  \end{frame}
  \begin{frame}{Caracter\'isticas do banco de dados}
        \begin{itemize}
            \item Possui 22 tabelas;
            \item Imagens s\~ao armazenadas no pr\'oprio banco de dados;
        \end{itemize}
  \end{frame}
  
  \section{Ferramentas Utilizadas}
  \begin{frame}{Linguagens de Programa\c{c}\~ao}
        \begin{itemize}
            \item Python;
            \item PHP, HTML e CSS.
        \end{itemize}
  \end{frame}
  
  \section{Resultados Obtidos}
  \begin{frame}{ Resultados }
        \begin{itemize}
            \item Foi criado o novo banco de dados;
            \item Foi criada a plataforma de acesso;
            \begin{itemize}
                \item Usu\'arios cadastrados podem fazer logIn no sistema;
                \item Usu\'arios e administradores, ap\'os realizar logIn, podem:
                \begin{itemize}
                    \item Cadastrar linhagens
                    \item Remover linhagens.
                \end{itemize}
                \item Administradores podem:
                \begin{itemize}
                    \item Cadastrar ou remover usu\'arios;
                    \item Visualizar as altera\c{c}\~oes realizadas por outros usu\'arios.
                \end{itemize}
            \end{itemize}
            \item Qualquer usu\'ario pode fazer buscas por n\'umero SDF.
        \end{itemize}
  \end{frame}
  
  \section{Problemas Encontrados}
  \begin{frame}{Problemas}
        \begin{itemize}
            \item O site n\~ao est\'a no ar.
            \item Dificuldades na inser\c{c}\~ao de imagens na p\'agina.
        \end{itemize}
  \end{frame}
  
  \section{Continua\c{c}\~ao do Trabalho}
  \begin{frame}{Continua\c{c}\~ao do Trabalho}
        \begin{itemize}
            \item Implementar busca por caracter\'isticas;
            \item Implementar a edi\c{c}\~ao de linhagens;
            \item Implementar ferramenta para desfazer altera\c{c}\~oes.
        \end{itemize}
  \end{frame}
\end{document}
