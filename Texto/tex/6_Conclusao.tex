\\\null \quad Após a construção do ambiente \acrshort{IoT} em escala reduzida, do sistema implementado seguindo os requisitos definidos durante a fase de planejamento do trabalho, da coleta de dados e da realização dos testes comparativos, é possível afirmar que os objetivos traçados no início do trabalho foram atingidos.
\\\null \quad O sistema implementado consegue calcular uma noção de qualidade razoável utilizando os dados coletados automaticamente e as informações fornecidas pelos usuários. A métrica utilizada foi validada por meio de comparação com outra ferramenta de medição de qualidade (\acrshort{OEE}) amplamente utilizada na indústria para medição de eficiência de equipamentos. A partir da pontuação disponibilizada pelo sistema, agentes (humanos ou máquinas) podem utilizar essas informações para que sejam tomadas decisões mais acertivas ao permitir que os dados coletados sejam considerados válidos.
\\\null \quad Os tópicos a seguir tratam de possibilidades para trabalhos futuros relacionados ao tema:
\begin{itemize}
  \item Implementação do sistema utilizando bancos de dados NOSQL;
  \item Associar as informações fornecidas pelos usuários aos registros dos sensores por meio de metadados;
  \item Implementação de uma ontologia para fornecer semântica aos dados;
  \item Permitir o envio de informações mais complexas aos sensores.
\end{itemize}
