\null \quad Este trabalho de conclusão de curso apresentou o desenvolvimento do sistema SenseHera, um sistema web colaborativo com o objetivo de armazenar, gerenciar e avaliar a qualidade de sensores em um ambiente \acrlong{IoT}.
\\\null \quad Para tanto, primeiramente, foi construído um ambiente \acrshort{IoT} em escala reduzida, composto de 5 (cinco) dispositivos sensitivos. Cada equipamento foi montado utilizando a placa Raspberry Pi 0 W e o sensor DHT11 (sensor de umidade e temperatura). Os sensores coletam as informações e, a cada minuto, as enviam para o sistema por meio de mensagens no formato JSON.
\\\null \quad Após a construção deste ambiente, o sistema web SenseHera foi desenvolvido seguindo como base os requisitos elaborados na fase de projeto, os quais estão elencados a seguir: funcionar em ambientes de baixa capacidade computacional; armazenar informações coletadas por sensores; apresentar as informações dos sensores de forma simplificada ao usuário; escalabilidade e facilidade para adição de sensores; permitir o envio de informações sobre fatos e sensações do ambiente pelo usuário e baseado nos dados coletados pelos sensores e informações enviadas pelos usuários, calcular uma nota para os dispositivos sensitivos, gerando uma noção de qualidade.
\\\null \quad Em sequência ao desenvolvimento do sistema web, foi realizada a coleta de dados por 95 (noventa e cinco) dias, o que proporcionou uma base de dados com 1.091.516 (um milhão noventa e um mil quinhentas e dezesseis) leituras. Durante o período de coleta de dados, os usuários colaboraram com o sistema, fornecendo perspectivas espaciais e psicológicas, as quais foram utilizadas, junto com os dados coletados por cada sensor, para realizar o cálculo de qualidade de cada dispositivo sensitivo.
 \\\null \quad Passados os 95 dias do início da coleta de dados, foi realizado um teste comparativo entre o resultado calculado pelo sistema SenseHera e a ferramenta \textit{\acrlong{OEE}} (\acrshort{OEE}).
 O sistema implementado consegue calcular uma noção de qualidade razoável utilizando os dados coletados automaticamente e as informações fornecidas pelos usuários. A métrica utilizada foi validada por meio de comparação com a \acrshort{OEE}, utilizada na indústria para medição de eficiência de equipamentos.
 \\\null\quad Ao disponibilizar a pontuação de cada sensor para os agentes (humanos ou máquinas), o sistema valida os sensores utilizados no ambiente \acrshort{IoT} no qual o SenseHera está inserido. Esses agentes, por sua vez, podem utilizar essas informações para que sejam tomadas decisões mais acertivas.
\\\null \quad Os temas a seguir tratam de possibilidades para trabalhos futuros: implementação do sistema utilizando bancos de dados NOSQL; associar as informações fornecidas pelos usuários aos registros dos sensores por meio de metadados; implementação de uma ontologia para fornecer semântica aos dados; permitir o envio de informações mais complexas aos sensores.
