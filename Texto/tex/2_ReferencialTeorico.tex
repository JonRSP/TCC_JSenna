O trabalho visa a obtenção e gerenciamento de metadados em informações provenientes de
dispositivos categorizados em \acrfull{IoT}. %expandiiiiiiiiir

\section{Sistemas Colaborativos}
\qqad A colaboração é um ponto central para o comportamento inteligente.
Atividades que envolvem colaboração permeiam praticamente todas as áreas da sociedade, estas atividades variam desde operações coordenadas,
bem planejadas e treinadas, como em times de esportes e grupos musicais, por exemplo, até colaborações espontâneas
entre pessoas que percebem que o problema em questão possui uma solução mais adequada se realizada em conjunto.
Tomando como exemplo a colaboração científica, esta ocorre em grande e pequena escala e entre diversas áreas da ciência.
 \cite{cbarbara}
\subsection{Colaboração versus Interação}
Segundo o dicionário, temos as seguintes definições:
\begin{itemize}
  \item \textbf{Colaboração:}
  \begin{enumerate}
    \item \textit{Trabalho feito em comum com um ou mais indivíduos;}
    \item \textit{Trabalho, ideia, doação etc. que contribui para a realização de algo ou para ajudar alguém.}
  \end{enumerate}
  \item \textbf{Interação:}
  \begin{enumerate}
    \item \textit{Ação mútua entre dois ou mais corpos ou indivíduos;}
    \item \textit{Comunicação entre indivíduos.}
  \end{enumerate}
\end{itemize}
 \null
\quad A partir destas definições, é possível notar que a colaboração consiste na união dos esforços
para a obtenção de um objetivo comum desejado por \textbf{todos} os indivíduos envolvidos, enquanto na interação
não há, necessariamente, a presença de um objetivo comum entre os envolvidos, tão pouco a união de esforços. \\ \null \quad
Para a autora Barbara Grosz \cite{cbarbara}, a colaboração, além de ser uma união de esforços para um objetivo comum,
precisa, também, ser projetada desde o início da resolução, não podendo ser apenas inserida durante o processo, visto que é
necessária a participação de todos os indivíduos para a modelagem do problema e, por consequência, a modelagem e execução da resolução.

\subsection{Sistemas Computacionais Colaborativos}
\quad Sistemas computacionais colaborativos são projetados para transformar o computador em não apenas uma ferramenta, mas em algo mais inteligente e útil.
Para o atingimento do objetivo desejado pela comunidade de usuários do sistema, estes devem auxiliar na alimentação de dados,
tornando o sistema mais inteligente, e por sua vez, conseguindo trazer informações mais consistentes e com maior relevância para os usuários.


\section{Internet das Coisas}%
\quad
A \acrlong{IoT} é um novo paradigma tecnológico idealizado como uma conexão global
de máquinas e dispositivos capazes de interagir entre si. A proposta de \acrshort{IoT} consiste em vários objetos do cotidiano trocando informações
mutuamente, através da internet, para serem mais eficientes e realizarem diversas tarefas.
Os objetos passam a agir de forma mais inteligente e sensorial, de modo a favorecer diversos setores como:
indústria, hospitais, agropecuária, transporte público e muitos outros. A partir desta
disponibilidade astronômica de recursos, a \acrshort{IoT} é reconhecida com uma das áreas mais importantes
em termos de tecnologia do futuro e está recebendo cada vez mais atenção de desenvolvedores, usuários e indústrias.
\\ \null
\quad
Um dos objetivos principais da \acrlong{IoT} é permitir
que humanos e máquinas possuam maior consciência de seus arredores.
 Esse maior entendimento do seu ambiente torna-se viável através da utilização
 de diversos tipos de dispositivos sensitivos (sensores) e, após a percepção
 de seu ambiente, é possível realizar ações por meio de dispositivos atuadores ou fazer análises. \\\\\\\\ \null
 \quad A \acrshort{IoT}
 inseriu um novo eixo em termos de comunicação, esta comunicação não é apenas a qualquer momento e em qualquer lugar,
 a \acrlong{IoT} permite que a comunicação seja realizada por qualquer coisa, conforme mostra a imagem \ref{refaxisiot}.
  \figura[!h]{axisiot.png}{Novo eixo introduzido em comunicações\cite{IOTS}}{refaxisiot}{scale=0.5}%
\\ \null
 \quad
 A \acrlong{IoT} surgiu a partir do cojunto de diferentes visões como podemos observar na \refFig{refiotvision}, cada qual com seus objetivos específicos.
 \figura[!h]{visionsiot}{O paradigma \acrlong{IoT} como um resultado de diferentes visões\cite{IOTS}}{refiotvision}{}%
\subsection{Definição}
	\quad Em 2012, a \acrfull{ITU} realizou estudos sobre infraestrutura
	de informação global, aspectos de procolos de internet e redes da próxima geração.
	A partir desse estudo foi construída a recomendação ITU-T Y.2060 \cite{ITU} que trata sobre a \acrlong{IoT}
	e possui o intuito de esclarecer o conceito e o escopo de \acrshort{IoT}, identificar
	as características fundamentais e os requerimentos de alto-nivel.
  \\ \null
  \quad	No documento produzido pela \acrshort{ITU}, foram consolidadadas as definições de:
	\begin{itemize}
		\item \acrlong{IoT}, "uma infraestrutura global para a Sociedade de Informações, permitindo serviços avançados ao
		interconectar (fisicamente e virtualmente) coisas devido à existência e evolução da interoperabilidade
	de tecnologias de comunicação e informação" \cite{ITU};
		\item Dispositivo, no contexto de \acrshort{IoT}, é um equipamento que, obrigatoriamente, possui a capacidade
		de comunicação e, opcionalmente, possui capacidade de sensitividade, atuação, captura de dados,
		armazenamento de dados e/ou processamento de dados \cite{ITU};
		\item Coisas, no contexto de \acrshort{IoT}, são "objetos
	no mundo físico (objetos físicos) ou no mundo das informações (objetos virtuais), os quais são capazes
	de serem identificados e integrados a uma rede de comunicações". Objetos físicos podem sentir, atuar e conectar.
	Objetos virtuais podem ser armazenados, processados e acessados.\cite{ITU}
	\end{itemize}
	\figura[!h]{iottec}{Visão geral técnica de \acrshort{IoT} \cite{ITU}}{refiottec}{}%
% [1] ITU\\
\subsection{Tecnologias essenciais}
	\subsubsection{\acrfull{RFID}}
		\begin{itemize}
			\item Esta tecnologia permite identificação automática e captura de informação por meio de rádio frequência.
			Dividem-se os dispositivos \acrshort{RFID} em duas grandes categorias: ativos e passivos. Dispositivos ativos dependem
			de uma fonte de energia constante para manter ativa e transmitir a informação. Dispositivos passivos não necessitam de energia constante,
			um campo eletromagnético energiza o dispositivo, o qual se torna apto a transferir a informação contida nele.
			\cite{refrfid}
		\end{itemize}
	\subsubsection{\acrfull{RSSF}}
		\begin{itemize}
			\item Esta tecnologia consiste na distribuição de dispositivos sensitivos autônomos para monitorar condições físicas ou
			ambientais e podem cooperar com sistemas \acrshort{RFID} para medir de forma mais eficaz localização, temperatura e movimentação, por exemplo.
			\cite{IOTS}
		\end{itemize}
	\subsubsection{Middleware}
		\begin{itemize}
			\item O middleware é a camada de abstração entre aplicações de software para facilitar para os desenvolvedores
			 realizar a comunicação entre softwares e operações de recebimento e envio de dados. O objetivo do middleware no contexto
			  de \acrshort{IoT} é simplificar a integração entre dispositivos heterogêneos e a camada de aplicação.
		\end{itemize}
	\subsubsection{Computação em nuvem}
		\begin{itemize}
			\item Computação em nuvem é um modelo para acesso de recursos compartilhados conforme a necessidade de um serviço. Um dos resultados mais notáveis
			da \acrshort{IoT} é a enorme quantidade de dados gerados por dispositivos conectados à internet \cite{IOTV}. A computação em nuvem é importante para o contexto de \acrlong{IoT}
			ao permitir um ambiente com alta escalabilidade.
		\end{itemize}
	\subsubsection{Aplicações de software}
		\begin{itemize}
			\item Aplicações \acrshort{IoT} permitem interações dispositivo-dispositivo e humano-dispositivo de uma forma confiável e robusta.
			As aplicações nos dispositivos devem garantir que as informações são recebidas e processadas de maneira adequada, no momento
			adequado.
		\end{itemize}

%\end{itemize}

\subsection{Desafios}
	\quad A \acrlong{IoT} possui diversos desafios devido à sua própria concepção, essas dificuldades devem ser ultrapassadas para que a
	\acrshort{IoT} possa ser amplamente e devidamente implantada. Alguns fatores críticos podem ser elencados:
	\subsubsection{Infraesturura de rede}
		\quad O custo para interconectar os dispositivos é alto. Para uma grande rede de sensores é necessário a distribuição de toda infraestrutua,
		de cabeamento ou infraestrutura sem fio.
	\subsubsection{Segurança}
		\quad Uma das principais dificuldades em um ambiente de \acrlong{IoT} é a segurança dos dados que são coletados pela grande quantidade de
		dispositivos sensitivos da rede. Uma invasão de um sistema crítico pode significar consequências graves em um ambiente automatizado.
  \subsubsection{Espaço de armazenamento}
	  \quad Um ambiente \acrshort{IoT} gera uma grande quantidade de dados, suponhamos um sistema de uma cidade inteligente que possui 10000 sensores
 	  de diversas categorias, cada sensor gerando uma mensagem de 10 kB a cada 5 minutos, por dia, temos um total de aproximadamente 29 GB de dados;
	  após um ano teríamos 10,5 TB de dados apenas para esta cidade.
  \subsubsection{Consumo de energia}
		\quad É imperativo que o consumo de energia dos dispositivos sensitivos seja o menor possível. Em muitos cenários é improvável a presença de uma
		rede de energia elétrica e até mesmo de manutenção constante, então os equipamentos devem conseguir se manter funcionais por meio de baterias por
		uma quantidade de tempo considerável.


%%%%%%%%%%%%%%%%%%%%%%%%%%%%%%%%%%%%%%%%%%%%%%%%%%%%%%%%%%%%%%%%%%%%%%%%%%%%%%%%
%%%%%%%%%%%%%%%%%%%%%%%%%%%%%%%%%%%%%%%%%%%%%%%%%%%%%%%%%%%%%%%%%%%%%%%%%%%%%%%%
%%%%%%%%%%%%%%%%%%%%%%%%%%%%%%%%%%%%%%%%%%%%%%%%%%%%%%%%%%%%%%%%%%%%%%%%%%%%%%%%
\section{Dados}%
	\quad Um ambiente de \acrlong{IoT} tem como objetivo a compreensão do ambiente em que está situado utilizando as informações geradas por diversos
  dispositivos sensitivos. Este entendimento é baseado em três tipos de dados \cite{SemIOT}:
  \begin{itemize}
    \item Dados gerados pelos dispositivos;
    \item Dados que descrevem os dispositivos;
    \item Dados que descrevem o ambiente.
  \end{itemize}
  \quad Normalmente, dispositivos \acrshort{IoT} tem sua semântica descrita em termos de suas capacidades sensitivas. A semântica do ambiente
  é determinada de acordo com o domínio da aplicação \cite{IOTdata}. Consequentemente, modelos de suporte à decisão são construídos
  baseados nos metadados que descrevem os dispositivos e seu ambiente.
  \\\\ \null
  \quad
  Os esforços de pesquisa em \acrlong{IoT} estão principalmente focados nos desafios de interoperabilidade, escalabilidade e integração entre dispositivos heterogêneos \cite{IOTdata},
  entretanto o desafio do dinamismo dos metadados em \acrshort{IoT} tem sido inexplorado \cite{collaborative}.

	\subsection{Metadados}
  \quad Metadados são as principais ferramentas para descrever e gerenciar recursos de informações extremamente dinâmicos, como os dados
  contidos na rede mundial de internet.
		\subsubsection{Princípios}
    \quad Os seguintes princípios são considerados as linhas de base para o desenvolvimento de soluções práticas em desafios de semântica
    e interoperabilidade de dispositivos em qualquer domínio e utilizando qualquer conjunto de metadados \cite{metadata}.
		\begin{itemize}
		  \item \textbf{Modularidade} de metadados é um pricípio chave para a organização de ambientes caracterizados pela diversidade de fontes e estilos
      de conteúdo e abordagens à descrição de recursos. O que permite que projetistas de esquemas de metadados criem novos esquemas baseados em projetos
      já existentes e se beneficiar de boas práticas já observadas. Em um ambiente com metadados modulares, elementos de informação de diferentes esquemas
      podem ser combinados de forma interoperável tanto sintaticamente quanto semanticamente.
      \item \textbf{Flexibilidade}. Sistemas de metadados precisam ser flexíveis para acomodar particularidade de uma determinada aplicação.
      Arquiteturas de metadados devem se adequar facilmente a noção de um esquema base aliado a elementos adicionais necessários para uma aplicação local ou um
      domínio específico sem comprometer a interoperabilidade proporcionada pelo esquema base.
      \item \textbf{Refinamento}. O nível de detalhes necessário para cada domínio de aplicação pode variar consideravelmente. Para evitar gastos
      desnecessários com armazenamento e processamento, o processo de desenvolvimento dos padrões de metadados devem permitir que os projetistas
      escolham o nível de detalhes apropriado para uma dada aplicação.
      \item \textbf{Multilinguismo}. O multilinguismo é essencial ao adotar arquiteturas de metadados que respeitem a diversidade linguística e cultural.
      Por ter a possibilidade de conectar sistemas de diversas partes do planeta, é importante que a comunicação dos metadados não tenham a linguagem e formatação
      como desafios a serem ultrapassados.
		\end{itemize}
  \subsection{Metadados de sensores}
    \quad Metadados de sensor é o modelo que descreve o sensor e suas capacidades, como por exemplo:
    \begin{itemize}
      \item Modelo do sensor;
      \item Localização do sensor;
      \item Unidade de medida utilizada;
      \item Grau de confiabilidade.
    \end{itemize}

     \quad A \acrfull{SensorML} \cite{SML} é uma coleção de padrões
    desenvolvida para representar informações de sensores em formato XML. O propósito da \acrshort{SensorML} é:
    \begin{itemize}
      \item Prover descrições de sensores e sistemas de sensores para gerenciamento de inventário;
      \item Providenciar informação sobre o sensor e sobre processamento;
      \item Auxiliar o processamento e análise de dados coletados por sensores;
      \item Suportar informações de geolocalização de valores coletados;
      \item Fornecer informações de desempenho;
      \item Providenciar uma descrição explícita sobre o processo em que os dados foram obtidos;
      \item Prover uma cadeia de processos executável para derivar novos produtos de informação;
      \item Arquivar propriedades fundamentais e suposições sobre os sistemas de sensores.
    \end{itemize}
\section{Plataforma}
\quad A plataforma escolhida para a implementação do ambiente \acrlong{IoT} é a placa Raspberry Pi 0 W. Para a implementação do servidor do sistema
será utilizada a placa Raspberry Pi modelo B+.
\subsection{Histórico}
\quad "A Fundação Raspberry Pi é uma instituição de caridade sediada no Reino Unido que trabalha para permitir que pessoas de todo o mundo tenham acesso
a computadores para que essas pessoas sejam capases de entender e dar forma ao mundo digital, sendo capazes de desvendar as soluções para os
problemas que as afligem e estando equipadas para os empregos do futuro" \cite{rasp}.\\\null
\quad Fundada em 2009 por David Braben, Jack Lang, Pete Lomas, Alan Mycroft, Robert Mullins e Eben Upton, com o apoio do Laboratório de computadores da Universidade de Cambridge e a empresa Broadcom
, a Fundação Raspberry Pi tem como objetivo promover e estudar a ciência da computação e assuntos correlatos, especialmente em nível escolar.
\\\null \quad Em 2011 a Fundação desenvolveu seu primeiro computador em placa única, nomeado Raspberry Pi. A meta seria vender estas placas
de desenvolvimento em duas versões, custando 25 dólares a versão mais simples e 35 dólares a versão mais elaborada. A versão mais complexa começou a ser vendida
em 29 de Fevereiro de 2012. O Raspberry Pi foi criado para estimular o estudo da ciência da computação em escolas.
\subsection{Características de Hardware}
\quad As placas escolhidas para a realização do trabalho possuem semelhanças e diferenças em suas montagens, como descrito nos tópicos a seguir:
\begin{itemize}
  \item Raspberry Pi modelo B+ \ref{figRaspB}:
  \begin{itemize}
    \item Tamanho: 85mm x 56mm;
    \item Processador ARM em clock de 700MHz;
    \item 512MB de memória RAM;
    \item 40 pinos GPIO;
    \item 4 portas USB 2.0.
    \figura[!h]{RaspB.png}{Placa de desenvolvimento Raspberry Pi modelo B+ \cite{RaspB}}{figRaspB}{scale=0.35}%
  \end{itemize}
  \item Raspberry Pi 0 W \ref{figrasp0}:
  \begin{itemize}
    \item Tamanho: 66mm x 30,5mm;
    \item Processador ARM em clock de 1GHz;
    \item 512MB de memória RAM;
    \item 40 pinos GPIO;
    \item 1 porta micro USB 2.0;
    \item 2.4GHz 802.11n wireless LAN;
    \item Bluetooth Classic 4.1 e Bluetooth LE.
    \figura[!h]{rasp0.png}{Placa de desenvolvimento Raspberry Pi 0 W \cite{rasp0}}{figrasp0}{scale=0.5}%
  \end{itemize}
\end{itemize}
\subsection{Características de Software}
\quad Em todas as placas será utilizado o sistema Raspbian Stretch Lite na última versão disponível (lançada em 27 de Junho de 2018).
Este sistema operacional baseado na distribuição linux Debian é a versão recomendada para todas as placas Raspberry Pi presentes no mercado.
