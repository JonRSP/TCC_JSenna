\null \quad Este capítulo apresenta o referencial teórico necessário para o desenvolvimento e compreensão deste projeto final. A Seção \ref{sec:siscol} trata de sistemas colaborativos, a Seção \ref{sec:internetofthings} apresenta informações sobre o domínio da \acrfull{IoT}, a Seção \ref{sec:dados} trata sobre dados no domínio de \acrshort{IoT}, a Seção \ref{sec:qualidade} trata da qualidade de dados e equipamentos e a Seção \ref{sec:plataforma} apresenta as plataformas utilizadas para construção do projeto.

\section{Sistemas Colaborativos}
\label{sec:siscol}
\null \quad A colaboração é um ponto central para qualquer comportamento inteligente.
Atividades que envolvem colaboração permeiam praticamente todas as áreas da sociedade; estas atividades variam desde operações coordenadas,
bem planejadas, como em times de esportes e grupos musicais até colaborações espontâneas
entre pessoas que percebem a necessidade de resolver problemas em conjunto.
Tomando como exemplo a colaboração para a produção do conhecimento científico, esta ocorre em vários níveis: entre pessoas ou até mesmo entre diversas áreas da ciência \cite{cbarbara}.

\subsection{Colaboração versus Interação}
\null \quad Segundo o dicionário Aurélio \cite{aurelio}, tem-se as seguintes definições:%referência do dicionário
\begin{itemize}
  \item \textbf{Colaboração:}
  \begin{enumerate}
    \item \textit{Trabalho feito em comum com um ou mais indivíduos;}
    \item \textit{Trabalho, ideia, doação etc. que contribui para a realização de algo ou para ajudar alguém.}
  \end{enumerate}
\newpage
  \item \textbf{Interação:}
  \begin{enumerate}
    \item \textit{Ação mútua entre dois ou mais corpos ou indivíduos;}
    \item \textit{Comunicação entre indivíduos.}
  \end{enumerate}
\end{itemize}
 \null
\quad A partir dessas definições, é possível concluir que a colaboração consiste na união dos esforços
para a obtenção de um objetivo comum desejado por todos os indivíduos envolvidos, enquanto na interação
não há, necessariamente, a presença de um objetivo comum entre os atores, tão pouco a união de esforços. \\ \null \quad
Para a autora Barbara Grosz \cite{cbarbara}, a colaboração, além de ser uma união de esforços para um objetivo comum,
precisa, também, ser projetada desde o início da resolução, não podendo ser apenas inserida durante o processo, visto que é
necessária a participação de todos os indivíduos para a modelagem do problema e, por consequência, a modelagem e execução da resolução.
\\
\null \quad Sistemas computacionais colaborativos são projetados para transformar um computador em algo mais complexo e útil do que apenas uma ferramenta.
Para atingir o objetivo desejado pela comunidade de usuários desses sistemas, seus integrantes devem auxiliar no povoamento de dados,
tornando o sistema mais inteligente, e, por sua vez, conseguindo produzir informações mais consistentes e relevantes para os usuários \cite{cbarbara}.
%aumentar este tópico

\section{Internet das Coisas}%
\label{sec:internetofthings}
\quad
A \acrlong{IoT} é uma nova abordagem tecnológica idealizada como uma conexão global
de máquinas e dispositivos capazes de interagir e até colaborar entre si. Em um ambiente \acrshort{IoT} vários objetos do cotidiano trocam informações
ou colaboram, por meio da internet, para serem mais eficientes e realizarem diversas tarefas.
Os objetos passam a agir de forma sensorial, de modo a favorecer diversos setores como:
indústria, hospitais, agropecuária, transporte público, entre muitos outros. A partir desta
elevada disponibilidade de recursos, a \acrshort{IoT} é reconhecida como uma das áreas tecnológicas mais importantes
do futuro e está recebendo, cada vez mais, atenção de desenvolvedores, usuários, governos e indústrias \cite{giusto}.
\\ \null
\quad
Um dos principais objetivos da \acrlong{IoT} é permitir
que humanos e máquinas aumentem sua capacidade de perceber, discriminar e avaliar dados e informações dos ambientes ao seu redor \cite{IOTS}.
 Esse melhor entendimento do seu ambiente torna-se viável por meio da utilização
 de diversos tipos de dispositivos sensitivos (sensores) para a aquisição de dados e, após a percepção e a interpretação
 de seu ambiente, é possível a realização de intervenções por meio de dispositivos atuadores \cite{IOTV}.  \\\null
 \quad A \acrshort{IoT}
 inseriu uma nova dimensão na comunicação, que não ocorre apenas a qualquer momento e em qualquer lugar, mas, também, pode ser realizada por qualquer objeto
 com capacidade comunicativa, conforme mostra a Figura \ref{refaxisiot}. Como pode ser observado, a comunicação pode ser realizada a qualquer momento, como por exemplo, à noite ou durante o dia; por qualquer agente, como um computador, um equipamento ou um usuário humano; e em qualquer lugar, como por exemplo, ao ar livre ou dentro de prédios.
  \figura[!h]{axisiot.png}{Nova dimensão introduzida em comunicações (adaptado de \cite{IOTS})}{refaxisiot}{height=250,width=\textwidth}%
\\ \null
 \quad
 A \acrlong{IoT} surgiu a partir de um cojunto de diferentes visões, como é possível observar na  \refFig{refiotvision}, cada qual com seus objetivos específicos. A \acrshort{IoT} pode ser definida como a intersecção entre a visão orientada a coisas, que trata, por exemplo, sobre objetos do dia a dia, dispositivos sensitivos e atuadores, a visão orientada à internet, que envolve tecnologias como por exemplo, IP para objetos inteligentes e Web das Coisas e a visão orientada à semântica, que trata, por exemplo, da racionalização de dados e tecnologias semânticas.
 \pagebreak
 \figura[!h]{visionsiot}{O paradigma \acrlong{IoT} como um resultado de diferentes visões (adaptado de \cite{IOTS})}{refiotvision}{height=325,width=\textwidth}%
\subsection{Definição de Internet das Coisas}
	\quad Em 2012, a \textit{\acrlong{ITU}} (\acrshort{ITU}) realizou estudos sobre infraestrutura
	de informação global, aspectos de protocolos de internet e redes da próxima geração.
	A partir desse estudo foi construída a recomendação ITU-T Y.2060 \cite{ITU} que trata sobre a \acrlong{IoT}
	e possui o intuito de definir seus conceito e escopo, identificar
	suas características fundamentais e seus requerimentos de alto-nivel.
  \\ \null
  \quad	No documento produzido pela \acrshort{ITU}, foram consolidadas as seguintes definições \cite{ITU}:
	\begin{itemize}
		\item \acrlong{IoT} é "uma infraestrutura global para a Sociedade de Informações, permitindo serviços avançados ao
		interconectar (fisicamente e virtualmente) coisas devido à existência e evolução da interoperabilidade
	de tecnologias de comunicação e informação" (tradução livre);%tradução livre
		\item Dispositivo, no contexto de \acrshort{IoT}, é um equipamento que, obrigatoriamente, possui a capacidade
		de comunicação e, opcionalmente, possui capacidade de sensitividade, atuação, captura de dados,
		armazenamento de dados e/ou processamento de dados;
		\item Coisas, no contexto de \acrshort{IoT}, são "objetos
	no mundo físico (objetos físicos) ou no mundo das informações (objetos virtuais), os quais são capazes
	de serem identificados e integrados a uma rede de comunicações"\ (tradução livre). Conforme mostra a Figura \ref{refiottec}, dispositivos comunicam-se entre si e objetos físicos, relacionados aos dispositivos, mapeam as informações coletadas em objetos virtuais. Objetos físicos podem sentir, atuar e conectar, ao passo que objetos virtuais podem ser armazenados, processados e acessados.
	\end{itemize}
	\figura[!h]{iottec}{Visão geral técnica de \acrshort{IoT} (adaptado de \cite{ITU})}{refiottec}{height=250,width=\textwidth}%
% [1] ITU\\
\subsection{Tecnologias Essenciais}
\quad Em geral, para que sistemas ou produtos baseados em \acrlong{IoT} possam ser bem sucedidos em grande escala, é necessário que algumas tecnologias sejam implementadas nessa rede \cite{IOTA}:
		\begin{itemize}
			\item \acrfull{RFID}: esta tecnologia permite identificação automática e captura de informação por meio de rádio frequência.
			Os dispositivos \acrshort{RFID} são divididos em duas grandes categorias: ativos e passivos. Os ativos dependem
			de uma fonte de energia constante para manter a informação ativa e para transmiti-la. Dispositivos passivos não necessitam de alimentação constante de energia.
			Estes funcionam a partir de sua energização por meio de um campo eletromagnético \cite{refrfid}.

		\end{itemize}
		\begin{itemize}
			\item \acrfull{RSSF}: esta tecnologia consiste na distribuição de dispositivos sensitivos autônomos para monitorar parâmetros físicos ou
			ambientais e podem cooperar com sistemas \acrshort{RFID} para adquirir dados de forma mais eficaz, tais como localização, temperatura e movimentação \cite{IOTS}.
		\end{itemize}
		\begin{itemize}
			\item \textit{Middleware}: o \textit{middleware} é a camada de abstração entre aplicações de software utilizada para auxiliar o trabalho dos desenvolvedores
			 no que toca à comunicação entre softwares e operações de recebimento e envio de dados. Seu objetivo, no contexto
			  de \acrshort{IoT}, é simplificar a integração entre dispositivos heterogêneos e a camada de aplicação \cite{middleware}.
		\end{itemize}
		\begin{itemize}
			\item Computação em nuvem: computação em nuvem é um modelo para acesso de recursos compartilhados conforme a necessidade de escalabilidade de um serviço. Um dos resultados mais característicos
			da \acrshort{IoT} é a enorme quantidade de dados gerados por dispositivos conectados à Internet \cite{IOTV}. A computação em nuvem é importante para o contexto de \acrlong{IoT}
			ao permitir um ambiente com alta adaptabilidade ao processamento de elevada quantidade de dados.
		\end{itemize}
		\begin{itemize}
			\item Aplicações de software: aplicações \acrshort{IoT} permitem interações e colaborações dispositivo-dispositivo e humano-dispositivo de uma forma confiável e robusta.
			As aplicações nos dispositivos devem garantir que as informações sejam recebidas e processadas de maneira adequada, no momento oportuno \cite{service}.
		\end{itemize}

%\end{itemize}

\subsection{Desafios da Internet das Coisas}
\label{subsec:desafios}
	\quad A \acrlong{IoT} possui diversos desafios conhecidos devido à sua própria concepção. Essas dificuldades devem ser superadas para que a
	\acrshort{IoT} possa ser devidamente implementada e amplamente utilizada. Alguns fatores críticos podem ser elencados \cite{roadmap}:
  \begin{itemize}
		\item Infraesturura de rede: o custo se eleva a partir da complexidade da infraestrutura para interconectar os dispositivos. Para uma grande rede de sensores é necessário o desdobramento da respectiva infraestrutura:
		 cabeamento, equipamentos de gerenciamento de redes, redes sem fio e contratação dos canais de comunicação, entre outros.

		\item Segurança: um dos principais desafios, em um ambiente de \acrlong{IoT}, é a manutenção de uma adequada segurança dos dados que são coletados pela grande quantidade de
		dispositivos sensitivos da rede. Uma invasão de um sistema crítico pode causar consequências graves em determinados ambientes automatizados.

	  \item Espaço de armazenamento: ao longo do tempo, um ambiente \acrshort{IoT} pode gerar uma grande quantidade de dados. A exemplo disso, um sistema de uma cidade inteligente que possui 10.000 sensores
 	  de diversas categorias, cada um deles gerando uma mensagem de 10 kB a cada 5 minutos, produz, por dia, um total de aproximadamente 29 GB de dados.
	  Após um ano, esse sistema produziria, aproximadamente, 10,5 TB de dados apenas para esta cidade.

		\item Consumo de energia: é imperativo que o consumo de energia dos dispositivos sensitivos seja o menor possível. Em muitos cenários é improvável a presença de uma
		rede de energia elétrica e até mesmo sua constante manutenção. Então, os equipamentos devem ter a capacidade de manter sua funcionalidade por meio do uso baterias por
		uma quantidade de tempo considerável.
\end{itemize}

%%%%%%%%%%%%%%%%%%%%%%%%%%%%%%%%%%%%%%%%%%%%%%%%%%%%%%%%%%%%%%%%%%%%%%%%%%%%%%%%
%%%%%%%%%%%%%%%%%%%%%%%%%%%%%%%%%%%%%%%%%%%%%%%%%%%%%%%%%%%%%%%%%%%%%%%%%%%%%%%%
%%%%%%%%%%%%%%%%%%%%%%%%%%%%%%%%%%%%%%%%%%%%%%%%%%%%%%%%%%%%%%%%%%%%%%%%%%%%%%%%
\section{Dados}%
\label{sec:dados}
	\quad Uma rede \acrlong{IoT} tem como objetivo a compreensão do ambiente em que está situado por meio das informações geradas por diversos
  sensores. Este entendimento é baseado em três tipos de dados \cite{SemIOT}:
  \begin{itemize}
    \item Dados gerados pelos dispositivos;
    \item Dados que descrevem os dispositivos;
    \item Dados que descrevem o ambiente.
  \end{itemize}
  \quad Normalmente, os dispositivos \acrshort{IoT} tem sua semântica descrita em termos de suas capacidades sensitivas. A semântica do ambiente
  é determinada de acordo com o domínio da aplicação \cite{IOTdata}. Consequentemente, modelos de suporte à decisão são construídos
  baseados nos metadados que descrevem os dispositivos e seu ambiente.
  \\\\ \null
  \quad
  Os esforços de pesquisa em \acrlong{IoT} estão principalmente focados nos desafios de interoperabilidade, escalabilidade e integração entre dispositivos heterogêneos \cite{IOTdata}.
  Entretanto, existe um desafio pouco explorado em lidar com o dinamismo dos metadados em \acrshort{IoT} \cite{collaborative}.

	\subsection{Metadados}
  \quad Metadados são as principais ferramentas para descrever e gerenciar recursos de informações  dinâmicos, como os dados contidos na Rede Mundial de Computadores, a \textit{World Wide Web} (WWW).
		\subsubsection{Princípios}
    \quad Os seguintes princípios são considerados básicos para o desenvolvimento de soluções práticas em desafios de semântica
    e interoperabilidade de dispositivos em qualquer domínio e em qualquer conjunto de metadados \cite{metadata}:
		\begin{itemize}
		  \item Modularidade: a modularidade de metadados é um princípio chave para a organização de ambientes caracterizados pela diversidade de fontes e estilos
      de conteúdo. A modularidade permite que projetistas criem novos esquemas de metadados baseados em projetos
      já existentes e se beneficiem de boas práticas já observadas. Em um ambiente com metadados modulares, elementos de informação de diferentes esquemas
      podem ser combinados de forma interoperável, tanto sintaticamente quanto semanticamente.
      \item Flexibilidade: sistemas de metadados precisam ser flexíveis para acomodar as particularidades de uma determinada aplicação.
      Arquiteturas de metadados devem se adequar facilmente à noção de um esquema base aliado a elementos adicionais necessários para uma aplicação local, ou a um
      domínio específico sem comprometer a interoperabilidade proporcionada pelo esquema base.
      \item Refinamento: o nível de detalhamento necessário para cada domínio de aplicação pode variar consideravelmente. Para evitar custos
      desnecessários com armazenamento e processamento, o desenvolvimento dos padrões de metadados devem permitir que os projetistas
      escolham o nível de detalhamento apropriado para uma dada aplicação.
      \item Multilinguismo: o multilinguismo é essencial ao adotar arquiteturas de metadados que respeitem a diversidade linguística e cultural.
      Por ter a possibilidade de conectar sistemas de diversas partes do planeta, é importante que a comunicação dos metadados respeite requisitos gerais de linguagem e formatação.
		\end{itemize}
  \subsection{Metadados de Sensores}
    \quad Denominam-se metadados de sensor o modelo que descreve o dispositivo sensitivo e suas capacidades, como por exemplo \cite{SML}:
    \begin{itemize}
      \item Modelo;
      \item Localização;
      \item Unidade de medida utilizada;
      \item Grau de confiabilidade.
    \end{itemize}

     \quad A \acrfull{SensorML} \cite{SML} é uma coleção de padrões
    desenvolvida para representar informações de sensores em formato XML, apesar de sua formatação obsoleta, os princípios utilizados para sua criação podem ser levados em consideração na utilização de outros padrões. Os propósitos da \acrshort{SensorML} incluem:
    \begin{itemize}
      \item Prover descrições de sensores e sistemas de sensores para gerenciamento de inventário;
      \item Providenciar informação sobre o sensor e sobre o processamento;
      \item Auxiliar o processamento e análise de dados coletados por sensores;
      \item Suportar informações de geolocalização de valores coletados;
      \item Fornecer informações de desempenho;
      \item Providenciar uma descrição explícita sobre o processo em que os dados foram obtidos;
      \item Prover uma cadeia de processos executáveis para derivar novos produtos de informação;
      \item Arquivar propriedades fundamentais e suposições sobre os sistemas de sensores.
    \end{itemize}
\section{Qualidade}
\label{sec:qualidade}
\quad Em ambientes \acrshort{IoT} as informações geradas são heterogêneas bem como suas fontes heterogêneas (diferentes classes e tipos de sensores), as quais não necessariamente são conhecidas ou podem ser auditadas \cite{dataquality}. O aumento no volume de informações geradas por ambientes \acrshort{IoT} causa uma preocupação com a qualidade dos dados coletados para que decisões mais acertivas possam ser tomadas por sistemas autônomos.
\\\null\quad Dados de alta qualidade são caracterizados pela consistência; se as informações geradas por fontes novas podem ser consideradas consistentes com as informações coletadas por fontes já verificadas, então os novos dados podem ser considerados de alta qualidade \cite{dataquality}.
\\\null\quad No contexto industrial, a qualidade de um equipamento é geralmente calculada por meio da ferramenta \textit{\acrlong{OEE}} (\acrshort{OEE}), criada em 1988 no Japão. Essa métrica utiliza os conceitos de Produtividade, Qualidade e Disponibilidade de um equipamento para definir sua qualidade geral. A qualidade geral, por sua vez, pode ser utilizada para se obter o nível de confiança de um dado equipamento, tomar decisões sobre substituição de equipamentos ou realizar mudanças no processo produtivo \cite{artigoOEE}.

\section{Plataforma de Hardware}
\label{sec:plataforma}
\quad Este projeto inclui o desenvolvimento de hardware (na montagem dos sensores) e software (na construção do sistema colaborativo).
\\\null\quad A plataforma de hardware escolhida para a implementação do ambiente \acrlong{IoT}, deste trabalho, é a placa Raspberry Pi 0 W. Para a implementação do servidor do sistema foi utilizado um serviço online da Amazon \cite{lightsail}.
\subsection{Histórico}
\quad Fundada em 2009 por David Braben, Jack Lang, Pete Lomas, Alan Mycroft, Robert Mullins e Eben Upton, com o apoio do Laboratório de computadores da Universidade de Cambridge e a empresa Broadcom, a Fundação Raspberry Pi tem como objetivo promover e estudar a ciência da computação e assuntos correlatos, especialmente em nível escolar.
\\\null\quad Segundo a Fundação Raspberry Pi \cite{rasp}, a organização "é uma instituição de caridade sediada no Reino Unido que trabalha para o empoderamento digital de pessoas ao redor do mundo,
de forma a torná-las capazes de entender e dar forma ao mundo digital. Desvendando as soluções para os
problemas que as afligem e estando, também, capazes para os empregos do futuro"\ (tradução livre).
\\\null \quad Em 2011, a Fundação desenvolveu seu primeiro computador em placa única, nomeado Raspberry Pi. A meta seria vender estas placas
de desenvolvimento em duas versões, custando 25 dólares a versão mais simples e 35 dólares a versão mais elaborada. A versão mais complexa começou a ser vendida
em 29 de Fevereiro de 2012. O Raspberry Pi foi criado para estimular o estudo da ciência da computação em escolas \cite{rasp}.
\subsection{Características de Hardware}
\quad A placa escolhida para a montagem dos sensores do trabalho e o servidor contratado possuem semelhanças e diferenças em suas características, como descrito nos tópicos a seguir:
\begin{itemize}
  \item Servidor Amazon Lightsail \cite{lightsail}:
  \begin{itemize}
    \item Processador de um núcleo;
    \item 512MB de memória RAM;
    \item 20 GB de SSD de armazenamento.
  \end{itemize}
  %\pagebreak
  \item Raspberry Pi 0 W (Figura \ref{figrasp0}):
  \begin{itemize}
    \item Tamanho: 66mm x 30,5mm;
    \item Processador ARM em clock de 1GHz;
    \item 512MB de memória RAM;
    \item 40 pinos GPIO;
    \item 1 porta micro USB 2.0;
    \item 2.4GHz 802.11n wireless LAN;
    \item Bluetooth Classic 4.1 e Bluetooth LE.
    \figura[!h]{rasp0.png}{Placa de desenvolvimento Raspberry Pi 0 W \cite{rasp0}}{figrasp0}{scale=0.5}%
  \end{itemize}
\end{itemize}
\subsection{Características de Software}
\quad Em todos os sensores foi utilizado o sistema Raspbian Stretch Lite na última versão disponível ao momento do desenvolvimento do projeto (lançada em 27 de Junho de 2018).
Este sistema operacional, baseado na distribuição linux Debian, é a versão recomendada pelo fabricante para todas as placas Raspberry Pi presentes no mercado.
\\\null\quad Para o desenvolvimento deste projeto foi contratado um servidor Amazon Lightsail \cite{lightsail}, nesta máquina virtual, foi instalado o sistema operacional Ubuntu 18.04. É utilizado o gerenciador de requisições HTTP Apache e o \textit{framework} Django.
\\\null\quad O Django é um \textit{framework} para desenvolvimento web baseado na linguagem Python; segundo seus desenvolvedores, é uma ferramenta que facilita o desenvolvimento ao simplificar as tarefas necessárias para assegurar o funcionamento online, segurança e escalabilidade \cite{django}.
