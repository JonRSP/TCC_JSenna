
\quad Este capítulo trata sobre os resultados obtidos após a execução do trabalho. A Seção \ref{sec:provadeconceito} trata de uma prova de conceito do sistema. A Seção \ref{sec:testecomparativo} refere-se a um teste comparativo realizado entre os resultados obtidos pelo sistema SenseHera e os resultados obtidos a partir de uma ferramenta muito utilizada na indústria para avaliação de qualidade de equipamentos.
%
\section{Prova de Conceito}
\label{sec:provadeconceito}
\\\null \quad Para a prova de conceito do sistema SenseHera, foi realizada a verificação da adição de um novo dispositivo sensitivo no ambiente \acrshort{IoT} em escala reduzida (Seção \ref{subsec:dispositivo}), inspeção dos dados coletados por um dos sensores durante um período de tempo (Seção \ref{subsec:dados}), realização de uma simulação de uma colaboração por parte de um usuário (Seção \ref{subsec:colaboracao}) e da utilização do sistema para a visualização dos dados coletados (Seção \ref{subsec:utilizacao}).

\subsection{Adição de Dispositivos no Ambiente IoT}
\label{subsec:dispositivo}
\\\null \quad Supõe-se a construção de um novo dispositivo sensitivo, como mostra a Figura \ref{prova1}, e sua configuração prévia (instalação do sistema operacional e das ferramentas necessárias para seu funcionamento e o \textit{download} do código desenvolvido para os sensores).
\pagebreak
\figura[!h]{prova1.jpg}{Novo dispositivo sensitivo construído}{prova1}{scale=0.17}
\\\null \quad Ao iniciar o código desenvolvido, o dispositivo cria um banco de dados local, realiza as leituras por meio do sensor DHT11, monta uma mensagem semelhante à mensagem mostrada na Figura \ref{mensagemJSONExemplo}, podendo ser diferente apenas no valor das leituras.
\\\null \quad Quando o SenseHera recebe a mensagem, é criada uma instância da entidade Sensor, as leituras são armazenadas no banco de dados do servidor e o identificador da instância criada é enviada ao sensor. A partir deste momento, já é possível observar na página web que um novo sensor foi adicionado com o nome de "Descrição temporária sensor 2", conforme mostra a Figura \ref{prova2}. Novas leituras desse dispositivo são associadas à instância cuja descrição é "Descrição temporária sensor 2". Esta descrição pode ser alterada utilizando a ferramenta de adminstração disponibilizada pelo Django.
\pagebreak
\figura[!h]{prova2.png}{Novo dispositivo sensitivo adicionado ao sistema}{prova2}{width=\textwidth, height=200}

\subsection{Inspeção dos Dados Coletados}
\label{subsec:dados}
\\\null \quad É realizado o acesso do servidor via \textit{\acrlong{SSH}} (\acrshort{SSH}) e o acesso direto ao banco de dados é feito a partir da ferramenta para linha de comando do PostgreSQL, conforme mostra a Figura \ref{prova3}.
\figura[!h]{prova3.png}{Acesso ao banco de dados via SSH}{prova3}{width=\textwidth,height=125}
\\\null \quad A Figura \ref{prova4} mostra as leituras coletadas durante um período de dez minutos pelo sensor cujo identificador é 4. Foram realizadas 22 leituras, 11 referentes à temperatura e 11 referentes à umidade. É possível observar a variação nos registros, o que mostra a variação real ocorrida no ambiente.
\figura[!h]{prova4.png}{Registros de sensor em um período de dez minutos}{prova4}{scale=0.8}

\subsection{Simulação de Colaboração Por Um Usuário}
\label{subsec:colaboracao}
\\\null \quad Supõe-se que um usuário do sistema, ao se deparar com algum dos QR-Codes apresentados na Figura \ref{prova5} decide escaneá-lo.
\pagebreak
\figura[!h]{prova5.jpg}{QR-Codes localizados próximos aos sensores}{prova5}{width=\textwidth, height=300}

\\\null \quad Ao ser redirecionado pelo \textit{software} de escaneamento para a página web, uma pergunta é aleatoriamente escolhida pelo sistema, a qual é apresentada para o usuário, conforme a Figura \ref{prova6}. Como o conjunto de respostas possíveis para uma pergunta é pré-definido, evitam-se erros de ortografia e erros de entendimento da pergunta, como mostra a Figura \ref{prova7}. O colaborador envia sua resposta para o SenseHera, a qual é armazenada no banco de dados do sistema, como mostra a Figura \ref{prova12}. A cada hora o sistema verifica se há uma concentração das respostas e realiza as operações referentes ao cálculo da pontuação do sensor.
\pagebreak
\figura[!h]{prova6.jpg}{Exemplo de pergunta apresentada ao colaborador}{prova6}{scale=0.22}
\figura[!h]{prova7.jpg}{Exemplo de um conjunto de respostas possíves apresentadas ao colaborador}{prova7}{scale=0.22}
\figura[!h]{prova12.png}{Respostas enviadas pelo usuário colaborador}{prova12}{scale=0.6}

\subsection{Utilização do Sistema}
\label{subsec:utilizacao}
\\\null \quad Supõe-se um usuário do sistema com interesse em controlar as condições de temperatura e umidade de um dos cômodos da residência em que o SenseHera está funcionando. Ao abrir a página de dados, todos os dispositivos sensitivos estão listados à esquerda, como mostra a Figura \ref{prova8}.

\figura[!h]{prova8.png}{Lista de dispositivos}{prova8}{width=\textwidth, height=275}

\\\null \quad O usuário, então, escolhe o cômodo que deseja visualizar as informações coletadas. Na página com os detalhes de um sensor, é possível ver informações sobre as últimas leituras e a pontuação do sensor, como mostra a Figura \ref{prova9}. Também é possível verificar a média das leituras nas últimas 24 horas, bem como a média histórica dos registros, respectivamente, nas Figuras \ref{prova10} e \ref{prova11}.
\figura[!h]{prova9.png}{Últimas leituras e pontuação do sensor}{prova9}{scale=0.5}
\newpage
\figura[!h]{prova10.png}{Gráficos com a média das leituras das últimas 24 horas}{prova10}{width=\textwidth, height=275}

\newpage

\figura[!h]{prova11.png}{Gráficos com a média histórica das leituras}{prova11}{width=\textwidth, height=275}

\\\null \quad Com as informações contidas na página de detalhes, o usuário pode tomar decisões mais acertivas sobre o que irá fazer para controlar, da forma desejada, as condições de temperatura e umidade do cômodo escolhido.


 \section{Teste Comparativo}
 \label{sec:testecomparativo}
\\\null \quad Os resultados deste projeto, além da própria implementação do sistema, são os dados coletados pelos sensores conectados ao ambiente \acrshort{IoT} e análises referentes a esses dados.
 Do momento em que foi executada a última restauração do banco de dados para seu estado inicial (realizada em 12/02/2019), até o dia 17/05/2019, foram coletadas 1.091.516 (um milhão noventa e um mil quinhentas e dezesseis) leituras.
\\\null \quad Para a comparação com a métrica de qualidade dos sensores implementada no sistema, foi utilizada a \textit{\acrlong{OEE}} (\acrshort{OEE}), uma métrica quantitativa desenvolvida em 1988 no Japão dentro do conceito de manutenção produtiva total que tem como objetivo medir diferentes tipos de perda de eficiência de um equipamento e indicar áreas para o aperfeiçoamento do processo de produção \cite{artigoOEE}.
\\\null \quad A Disponibilidade (D) de um equipamento pode ser dada pela razão entre o tempo útil (U) de funcionamento e o tempo total de funcionamento (N), como mostra a Equação \ref{eq:disp}. No escopo deste trabalho, o tempo útil é a quantidade de dias com o número de registros armazenados maior do que a metade do total esperado. O número de registros esperados por dia corresponde à multiplicação da quantidade de leituras por hora e o número de horas em um dia, o que corresponde a 2880 registros diários. A Tabela \ref{disponibilidade} mostra os dados utilizados para o cálculo de disponibilidade dos sensores.

\begin{equation}
  D = \frac{U}{N}
  \label{eq:disp}
\end{equation}

\begin{center}
\captionof{table}{Tabela que representa a disponibilidade dos sensores.}\label{disponibilidade}
\resizebox{\textwidth}{!}{%
\begin{tabular}{rrrrr}
  \hline
  Sensor     & \begin{tabular}[c]{@{}l@{}}Número de dias\\  em funcionamento (N)\end{tabular} & \begin{tabular}[c]{@{}l@{}}Tempo útil\\em dias (U)\end{tabular} & \begin{tabular}[c]{@{}l@{}}Tempo de\\  não funcionamento\\em dias (I)\end{tabular} & Disponibilidade (D) \\
\hline
Oficina    & 53                                                                             & 49                 & 4                                                                     & 92,45\%             \\
\hline
Escritório & 95                                                                            & 89                 & 6                                                                     & 93,68\%             \\
\hline
Quarto     & 95                                                                            & 93                 & 2                                                                     & 97,89\%             \\
\hline
Varanda    & 95                                                                            & 87                 & 8                                                                     & 91,57\%             \\
\hline
Piano      & 95                                                                            & 89                 & 6                                                                     & 93,68\%

\end{tabular}%
}
\end{center}


\\\null \quad A Produtividade (P) de um equipamento pode ser calculada pela razão entre a quantidade de unidades produzidas (L) e a quantidade esperada (E), como mostra a Equação \ref{eq:prod}. No escopo desse trabalho, é utilizada como unidade produzida a leitura enviada pelo sensor e armazenada no servidor. O número de leituras esperado (E) é dado pelo número de dias de funcionamento (N) multiplicado pela quantidade de registros que deveriam ser enviados por dia (número de leituras por hora vezes 24 horas), no contexto dos sensores utilizados no ambiente \acrshort{IoT} em escala reduzida, cada dispositivo sensitivo envia duas leituras por minuto, uma referente à temperatura e outra à umidade,  seguindo a Equação \ref{eq:esp}. A Tabela \ref{produtividade} mostra os dados utilizados para o cálculo da produtividade dos sensores.

\begin{equation}
  P = \frac{L}{E}
  \label{eq:prod}
\end{equation}

\begin{equation}
  E = N * (60 * 2 * 24)
  \label{eq:esp}
\end{equation}

\begin{center}
\captionof{table}{Tabela que representa a produtividade dos sensores.}\label{produtividade}
\resizebox{\textwidth}{!}{%
\begin{tabular}{lrrrr}
  \hline
Sensor     & \multicolumn{1}{l}{\begin{tabular}[c]{@{}l@{}}Número de dias\\  em funcionamento (N)\end{tabular}} & \multicolumn{1}{l}{Número de leituras (L)} & \multicolumn{1}{l}{\begin{tabular}[c]{@{}l@{}}Número de leituras\\  esperado (E)\end{tabular}} & \multicolumn{1}{l}{Disponibilidade (D)} \\
\hline
Oficina    & 53                                                                                                 & 140754                                     & 152640                                                                                         & 92,21\%                                 \\
\hline
Escritório & 95                                                                                                 & 233500                                     & 273600                                                                                         & 85,34\%                                 \\
\hline
Quarto     & 95                                                                                                 & 246550                                     & 273600                                                                                         & 90,11\%                                 \\
\hline
Varanda    & 95                                                                                                 & 233722                                     & 273600                                                                                         & 85,42\%                                 \\
\hline
Piano      & 95                                                                                                 & 236990                                     & 273600                                                                                         & 86,62\%
\end{tabular}%
}
\end{center}

\\\null \quad A característica Qualidade (Q) em uma avaliação de \textit{\acrlong{OEE}} de um equipamento é dada pela razão entre a quantidade total produzida (T) deduzida da quantidade inutilizada ou retrabalhada (U) e a quantidade total produzida (T), conforme mostra a Equação \ref{eq:qualidade}.
\\ \null \quad Visto que neste trabalho de conclusão de curso o objetivo é apresentar uma avaliação dos sensores, não foi realizada uma avaliação do dado produzido pelos equipamentos, não exitindo, então, a noção de quantidade inutilizada ou retrabalhada uma vez que uma leitura pode apenas ser armazenada ou não, o que leva a qualidade do que é produzido ser sempre igual a 100\%.

\begin{equation}
  Q = \frac{T-U}{T}
  \label{eq:qualidade}
\end{equation}


\\\null \quad A \acrshort{OEE} é dada pela multiplicação entre os valores de Produtividade, Disponibilidade e Qualidade, conforme a Equação \ref{eq:OEE}, gerando a Tabela \ref{tabelaOEE} a seguir.

\begin{equation}
  OEE = P * D * Q
  \label{eq:OEE}
\end{equation}

\begin{center}
\captionof{table}{Tabela que representa a \acrshort{OEE} dos sensores.}\label{tabelaOEE}
\resizebox{\textwidth}{!}{%
\begin{tabular}{lrrrr}
  \hline
Sensor     & \begin{tabular}[c]{@{}l@{}}Disponibilidade (D)\end{tabular} & Produtividade (P) & \begin{tabular}[c]{@{}l@{}}Qualidade (Q)\end{tabular} & OEE \\
\hline
Oficina    & 92,45\%                                                                             & 92,21\%                 & 100\%                                                                     & 85,24\%             \\
\hline
Escritório & 93,68\%                                                                            & 85,34\%                 & 100\%                                                                     & 79,94\%             \\
\hline
Quarto     & 97,89\%                                                                            & 90,11\%                 & 100\%                                                                     & 88,20\%             \\
\hline
Varanda    & 91,57\%                                                                            & 85,42\%                & 100\%                                                                     & 78,22\%             \\
\hline
Piano      & 93,68\%                                                                            & 86,62\%                & 100\%                                                                     & 81,15\%
\end{tabular}%
}
\end{center}

\\\null \quad O sistema de pontuação utilizado está descrito na Seção \ref{subsec:pontuacao}, o que inclui, por exemplo, as considerações de regularidade das leituras, interrupções no envio de dados e as colaborações dos usuários.
\\\null \quad A Tabela \ref{tabelaComparacao} mostra a média da pontuação dos sensores, calculada pelo sistema em comparação com a pontuação \acrshort{OEE}. A utilização da média da pontuação calculada foi necessária devido ao fato da \textit{\acrlong{OEE}} tratar de dados históricos.

\begin{center}
\captionof{table}{Tabela que representa a comparação entre \acrshort{OEE} dos sensores e a pontuação calculada pelo sistema.}\label{tabelaComparacao}
\resizebox{200}{!}{%
\begin{tabular}{lrr}
  \hline
Sensor     & \multicolumn{1}{l}{OEE} & \multicolumn{1}{l}{\begin{tabular}[c]{@{}l@{}}Pontuação\\ Calculada\end{tabular}} \\
\hline
Oficina    & 85,24\%                 & 88\%                                                                              \\
\hline
Escritório & 79,94\%                 & 84,4\%                                                                            \\
\hline
Quarto     & 88,20\%                 & 89\%                                                                              \\
\hline
Varanda    & 78,22\%                 & 83,6\%                                                                            \\
\hline
Piano      & 81,15\%                 & 85,7\%

\end{tabular}%
}
\end{center}
