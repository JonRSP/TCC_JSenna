
\quad Este capítulo trata sobre os resultados obtidos após a execução do trabalho.

\section{Resultados}
\\\null \quad Os resultados deste projeto, além da própria implementação do sistema, são os dados coletados pelos sensores conectados ao ambiente \acrshort{IoT} e análises referentes a esses dados.
 Do momento em que foi executada a última restauração do banco de dados para seu estado inicial (realizada em 12/02/2019) até o momento da escrita deste documento foram coletadas 1.091.516 (um milhão noventa e um mil quinhetas e dezesseis) leituras.
\\\null \quad Para a comparação com a métrica de qualidade dos sensores implementada no sistema, foi utilizado a \textit{\acrlong{OEE}} (\acrshort{OEE}), uma métrica quantitativa desenvolvida em 1988 no Japão dentro do conceito de manutenção produtiva total que tem como objetivo medir diferentes tipos de perda de eficiência de um equipamento e indicar áreas para o aperfeiçoamento do processo de produção \cite{artigoOEE}.
\\\null \quad A Disponibilidade (D) de um equipamento pode ser dada pela razão entre o tempo útil (U) de funcionamento e o tempo total de funcionamento (N), como mostra a Equação \ref{eq:disp}. No escopo deste trabalho o tempo útil é a quantidade de dias com o número de registros armazenados maior do que a metade do total esperado. O número de registros esperados por dia corresponde à multiplicação da quantidade de leituras por hora e o número de horas em um dia, o que corresponde a 2880 registros diários. A Tabela \ref{disponibilidade} mostra os dados utilizados para o cálculo de disponibilidade dos sensores.

\begin{equation}
  D = \frac{U}{N}
  \label{eq:disp}
\end{equation}
\newpage
\begin{center}
\captionof{table}{Tabela que representa a disponibilidade dos sensores}\label{disponibilidade}
\resizebox{\textwidth}{!}{%
\begin{tabular}{lllll}
  \hline
  Sensor     & \begin{tabular}[c]{@{}l@{}}Número de dias\\  em funcionamento (N)\end{tabular} & \begin{tabular}[c]{@{}l@{}}Tempo útil\\em dias (U)\end{tabular} & \begin{tabular}[c]{@{}l@{}}Tempo de\\  não funcionamento\\em dias (I)\end{tabular} & Disponibilidade (D) \\
\hline
Oficina    & 53                                                                             & 49                 & 4                                                                     & 92,45\%             \\
\hline
Escritório & 101                                                                            & 95                 & 6                                                                     & 94,05\%             \\
\hline
Quarto     & 101                                                                            & 99                 & 2                                                                     & 98,01\%             \\
\hline
Varanda    & 101                                                                            & 93                 & 8                                                                     & 92,07\%             \\
\hline
Piano      & 101                                                                            & 95                 & 6                                                                     & 94,05\%

\end{tabular}%
}
\end{center}


\\\null \quad A Produtividade (P) de um equipamento pode ser calculada pela razão entre a quantidade de unidades produzidas (L) e a quantidade esperada (E), como mostra a Equação \ref{eq:prod}. No escopo desse trabalho é utilizada como unidade produzida a leitura enviada pelo sensor e armazenada no servidor. O número de leituras esperado (E) é dado pelo número de dias de funcionamento (N) multiplicado pela quantidade de registros que deveriam ser enviados por dia (número de leituras por hora vezes 24 horas), seguindo a Equação \ref{eq:esp}. A Tabela \ref{produtividade} mostra os dados utilizados para o cálculo da produtividade dos sensores.

\begin{equation}
  P = \frac{L}{E}
  \label{eq:prod}
\end{equation}

\begin{equation}
  E = N * (60 * 2 * 24)
  \label{eq:esp}
\end{equation}

\begin{center}
\captionof{table}{Tabela que representa a produtividade dos sensores}\label{produtividade}
\resizebox{\textwidth}{!}{%
\begin{tabular}{lllll}
  \hline
  Sensor     & \begin{tabular}[c]{@{}l@{}}Número de dias\\  em funcionamento (N)\end{tabular} & Número de leituras (L) & \begin{tabular}[c]{@{}l@{}}Número de leituras\\  esperado (E)\end{tabular} & Produtividade (P) \\
  \hline
  Oficina    & 53                                                                             & 140754                 & 152640                                                                     & 92,21\%             \\
  \hline
  Escritório & 101                                                                            & 233500                 & 290880                                                                     & 80,27\%             \\
  \hline
  Quarto     & 101                                                                            & 246550                 & 290880                                                                     & 84,76\%             \\
  \hline
  Varanda    & 101                                                                            & 233722                 & 290880                                                                     & 80,35\%             \\
  \hline
  Piano      & 101                                                                            & 236990                 & 290880                                                                     & 81,47\%
\end{tabular}%
}
\end{center}

\\\null \quad A Qualidade (Q) de um equipamento é dada pela razão entre a quantidade total produzida deduzida da quantidade inutilizada ou retrabalhada e a quantidade total produzida. No escopo deste trabalho não há a noção de quantidade inutilizada ou retrabalhada visto que uma leitura pode apenas ser armazenada ou não, o que leva a qualidade do que é produzido ser sempre igual a 100\%.
\\\null \quad A \acrshort{OEE} é dada pela multiplicação entre os valores de Produtividade, Disponibilidade e Qualidade, conforme a Equação \ref{eq:OEE}, gerando a Tabela \ref{tabelaOEE} a seguir.

\begin{equation}
  OEE = P * D * Q
  \label{eq:OEE}
\end{equation}

\begin{center}
\captionof{table}{Tabela que representa a \acrshort{OEE} dos sensores}\label{tabelaOEE}
\resizebox{\textwidth}{!}{%
\begin{tabular}{lllll}
  \hline
Sensor     & \begin{tabular}[c]{@{}l@{}}Disponibilidade (D)\end{tabular} & Produtividade (P) & \begin{tabular}[c]{@{}l@{}}Qualidade (Q)\end{tabular} & OEE \\
\hline
Oficina    & 92,45\%                                                                             & 92,21\%                 & 100\%                                                                     & 85,24\%             \\
\hline
Escritório & 94,05\%                                                                            & 80,27\%                 & 100\%                                                                     & 75,49\%             \\
\hline
Quarto     & 98,01\%                                                                            & 84,76\%                 & 100\%                                                                     & 83,07\%             \\
\hline
Varanda    & 92,07\%                                                                            & 80,35\%                & 100\%                                                                     & 73,97\%             \\
\hline
Piano      & 94,05\%                                                                            & 81,47\%                 & 100\%                                                                     & 76,62\%
\end{tabular}%
}
\end{center}

\\\null \quad O sistema de pontuação utilizado está descrito na seção \ref{subsec:pontuacao}, o que inclui, por exemplo, as considerações de regularidade das leituras, interrupções no envio de dados e as colaborações dos usuários.
\\\null \quad A Tabela \ref{tabelaComparacao} mostra a média da pontuação dos sensores calculada pelo sistema em comparação com a pontuação \acrshort{OEE}. A utilização da média da pontuação calculada foi necessária devido ao fato da \textit{\acrlong{OEE}} tratar de dados históricos.

\begin{center}
\captionof{table}{Tabela que representa a comparação entre \acrshort{OEE} dos sensores e a pontuação calculada pelo sistema}\label{tabelaComparacao}
\resizebox{250}{!}{%
\begin{tabular}{lllll}
  \hline
Sensor     & OEE  & \begin{tabular}[c]{@{}l@{}}Pontuação \\ Calculada \end{tabular}\\
\hline
Oficina    & 85,24\%                                                                             & 88\%                 \\
\hline
Escritório & 75,49\%                                                                            & 84,4\%                 \\
\hline
Quarto     & 83,07\%                                                                            & 89\%                 \\
\hline
Varanda    & 73,97\%                                                                            & 83,6\%                \\
\hline
Piano      & 76,62\%                                                                            & 85,7\%

\end{tabular}%
}
\end{center}
