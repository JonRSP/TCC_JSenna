\newcommand{\texCommand}[1]{\texttt{\textbackslash{#1}}}%

\newcommand{\exemplo}[1]{%
\vspace{\baselineskip}%
\noindent\fbox{\begin{minipage}{\textwidth}#1\end{minipage}}%
\\\vspace{\baselineskip}}%

\newcommand{\exemploVerbatim}[1]{%
\vspace{\baselineskip}%
\noindent\fbox{\begin{minipage}{\textwidth}%
#1\end{minipage}}%
\\\vspace{\baselineskip}}%


\quad Este capítulo descreve os métodos utilizados para a realização deste trabalho.

%%%%%%%%%%%%%%%%%%%%%%%%%%%%%%%%%%%%%%%%%%%%%%%%%%%%%%%%%%%%%%%%%%%%%%%%%%%%%%%%
%%%%%%%%%%%%%%%%%%%%%%%%%%%%%%%%%%%%%%%%%%%%%%%%%%%%%%%%%%%%%%%%%%%%%%%%%%%%%%%%
%%%%%%%%%%%%%%%%%%%%%%%%%%%%%%%%%%%%%%%%%%%%%%%%%%%%%%%%%%%%%%%%%%%%%%%%%%%%%%%%
\section{Revisão Sistemática}
%escrever sobre o porque de usar a revisão sistemática
\quad Foi definido um processo de revisão sistemática apenas para que fosse seguido um protocolo para a aquisição de referências bibliográficas. Esta revisão consiste em duas fases:
\begin{itemize}
  \item Planejar;
  \item Executar;
  % \item Documentar.
\end{itemize}
\quad A fase de planejamento constitui-se na especificação do protocolo.
A fase de execução representa a coleta dos dados de forma a atender as especificações
exigidas na fase de planejamento.
% A fase de documentação implica na consolidação dos dados obtidos.

\subsection{Planejamento}
\quad O objetivo desta revisão sistemática é a identificação de trabalhos acadêmicos
que expõem resultados, projeções, explicações ou elucidações sobre o tema de armazenamento ou gerenciamento
de metadados para \acrlong{IoT} de forma colaborativa ou não, com o propósito de que haja uma metodologia durante a construção das referêcias do presente trabalho.

\subsubsection{Questões de Estudo}
\quad Esta revisão tem como objetivo responder às seguintes questões:
\begin{itemize}
  \item Há estudos sobre o armazenamento e gerenciamento de metadados para \acrlong{IoT}?
  \item Quais são os métodos mais utilizados para o armazenamento e gerenciamento de metadados para \acrshort{IoT}?
  \item Há estudos sobre a utilização de abordagens colaborativas em \acrlong{IoT}?
  \item Em quais aplicações os metadados estão sendo utilizados nos trabalhos?
\end{itemize}

\subsubsection{Estratégia de Busca}
\quad Foi determinada uma estratégia para a realização das buscas nas bases de dados escolhidas conforme a seguir:

\paragraph{Definição da String de Busca}
\begin{itemize}
  \item \textbf{População}: Utilizou-se como tema principal metadados para \acrshort{IoT}. Para procurar, foram utilizadas as palavas-chave 'Internet of things metadata', 'IoT metadata' e 'Collaborative IoT';
  \item \textbf{Intervenção}: O foco é verificar middlewares, modelos e esquemas. Os termos utilizados para pesquisa foram 'middleware', 'management', 'model' e 'schema';
  \item \textbf{Comparação}: O foco deste trabalho não se limitou a estudos comparativos;
  \item \textbf{Resultado}: Tem-se como objetivo a procura de avaliações, definições, validações e implementações de middlewares e/ou esquemas utilizados em pesquisas científicas. Desta forma, foram obtidas as seguintes palavras-chave
  'validation', 'evaluation' e 'implementation'.
\end{itemize}

\quad A string de busca gerada é a seguinte:
(('Internet of things metadata') \textbf{OR} ('IoT metadata') \textbf{OR} ('Collaborative IoT')) \textbf{AND} (('middleware') \textbf{OR} ('schema') \textbf{OR} ('management') \textbf{OR} ('model')) \textbf{AND} (('validation') \textbf{OR} ('evaluation') \textbf{OR} ('implementation'))

\paragraph{Fontes de Busca}
\subparagraph{}
 Foram escolhidas as seguintes bases digitais para que as buscas sejam realizadas:
\begin{itemize}
  \item Google Acadêmico (https://scholar.google.com.br/)
  \item JSTOR (https://www.jstor.org/)
  \item Periódicos CAPES (https://www.periodicos.capes.gov.br/)
\end{itemize}
\quad Bases escolhidas devido a sua relevância e sua grande abrangência sobre diversos temas.
\pagebreak
\paragraph{Idioma}
\subparagraph{}
\quad O idioma de preferência para seleção de artigos será a língua inglesa, entretanto, trabalhos científicos escritos em
língua portuguesa não serão descartados, desde que atinjam os requisitos para inclusão.

\subsubsection{Seleção dos Estudos}
\paragraph{Critérios de Inclusão e Exclusão}
\subparagraph{Critérios de Inclusão}
\begin{itemize}
  \item O texto integral dos trabalhos devem estar disponíveis nas bases de dados escolhidas previamente;
  \item Serão consideradas apenas as publicações posteriores à 2006 (ano no qual esta área começou a ser pesquisada de forma mais intensa), salvo em casos de fontes relevantes que contenham definições necessárias para a realização deste trabalho;
  \item O trabalho deve fazer menção a "metadados em ambientes \acrlong{IoT}".
\end{itemize}

\subparagraph{Critérios de Exclusão}
\begin{itemize}
  \item Trabalhos publicados anteriormente à 2006, exceto fontes relevantes que contenham definições pertinentes para este estudo;
  \item Trabalhos parcialmente disponíveis nas bases de dados digitais escolhidas;
  \item Publicações cujo texto não trate do tema "metadados em ambientes \acrshort{IoT}";
  \item Trabalhos que não contenham propostas, comparações ou avaliações de métodos para o gerênciamento ou armazenamento de metadados;
  \item Trabalhos publicados em mútiplas bases de dados. O trabalho será contado apenas uma vez.
\end{itemize}

\subsection{Execução}
\subsubsection{Processo de Seleção dos Estudos}
\quad Os artigos obtidos por meio da estratégia acima descrita passarão por um processo de avaliação sistêmico, com base nos critérios anteriormente especificados.
Desta forma, os artigos que atingirem os parâmetros estabelecidos serão adicionados à base de estudos da revisão sistemática. A estratégia para a pesquisa e seleção é:
\begin{enumerate}
  \item Pesquisa de trabalhos científicos nas bases de dados definidas utilizando as strings de busca;
  \item Leitura do título, resumo, palavras chave e data de publicação, aplicando os critérios de inclusão e exclusão definidos;
  \item Leitura da introdução e conclusão dos trabalhos que foram mantidos na fase anterior;
  \item Os trabalhos resultantes serão lidos por completo, e as informações pertinentes serão coletadas.
\end{enumerate}

\subsubsection{Resultado da Seleção de Estudos}
\paragraph{Fase I - Total de estudos obtidos}
\subparagraph{}
\quad Utilizando todas as combinações possíveis da string de busca foram encontrados as seguintes quantidades de estudos sobre o tema:
\begin{itemize}
  \item Google Acadêmico - 40 estudos;
  \item JSTOR - 3 estudos;
  \item Periódicos CAPES - 25 estudos.
\end{itemize}
\paragraph{Fase II - Estudos que possuem relação com o tema}
\subparagraph{}
\quad Nesta fase realizou-se a leitura do título, resumo, palavras chave e identificação da data de publicação.
Aplicados os critérios estabelecidos, foi obtido o seguinte resultado:
\begin{itemize}
  \item Google Acadêmico - 24 estudos;
  \item JSTOR - 3 estudos;
  \item Periódicos CAPES - 13 estudos.
\end{itemize}
\paragraph{Fase III - Estudos que possuem relação com o tema}
\subparagraph{}
\quad Os artigos coletados passaram pela leitura da introdução e conclusão.
Aplicados os critérios dessa fase, o seguinte resultado foi obtido:
\begin{itemize}
  \item Google Acadêmico - 12 estudos;
  \item JSTOR - 1 estudo;
  \item Periódicos CAPES - 7 estudos.
\end{itemize}
\paragraph{Fase IV - Leitura dos Estudos}
\subparagraph{}
\quad O material que passou pela seleção das fases II e III foi lido integralmente e as
informações pertinentes serão incluídas nas seções adequadas.
\subsubsection{Respostas às Questões de Estudo}
\begin{itemize}
  \item Há estudos sobre o armazenamento de metadados para \acrlong{IoT}?
 \begin{itemize}
    \item \textit{Existem estudos sobre o armazenamento de metadados para \acrshort{IoT}, entretanto 4 deles tratam
    sobre este tema específico, representando um total de aproximadamente 20\% dos artigos considerados úteis para o trabalho.}
  \end{itemize}
  \item Quais são os métodos mais utilizados para o armazenamento de metadados para \acrshort{IoT}?
    \begin{itemize}
    \item \textit{Foi descoberto que não há um consenso sobre o método de armazenamento de metadados \acrshort{IoT}. Cada estudo introduz um método diferente.}
  \end{itemize}
  \item Há estudos sobre a utilização de abordagens colaborativas em \acrlong{IoT}?
  \begin{itemize}
    \item \textit{Há apenas um estudo que utiliza abordagens colaborativas para a obtenção de metadados para \acrshort{IoT}}
  \end{itemize}
  \item Em quais aplicações os metadados estão sendo utilizados nos trabalhos?
  \begin{itemize}
    \item \textit{As smart buildings (construções inteligentes) são as principais aplicações em que os metadados estão sendo utilizados em \acrshort{IoT}.}
  \end{itemize}
\end{itemize}

\subsubsection{Conclusão da Revisão Sistemática}
\quad Após a utilização da string de busca e das fases II e III, foi obtido um total de 20 estudos significativos,
ou seja, 29\% do total de estudos coletados. Estes dados demonstram uma possível ineficiência do método
de revisão sistemática quando se trata de assuntos novos ou pouco trabalhados.

\section{Construção do Ambiente IoT}
\quad O ambiente \acrshort{IoT} em escala reduzida que será utilizado nesta seção é essencial para o teste de conceito
do sistema proposto. A alimentação de dados pelos componentes da rede é necessária para que haja
dados suficientes para o processamento do programa a ser desenvolvido. As informações geradas por esses dispositivos, após serem conectados ao sistema, em colaboração
com os usuários, permitirão o pleno funcionamento da proposta.
\subsection{Dispositivos Sensitivos}
\quad Foram construídos 5 dispositivos sensitivos utilizando a plataforma Raspberry Pi 0 W seguindo o esquema da figura \ref{raspsensor}. Esta placa
foi escolhida por seu baixo valor de custo, seu bom desempenho computacional, sua capacidade de conexão wireless disponível diretamente na placa, sem necessidade de equipamentos extras e a possibilidade de executar um sistema operacional baseado em linux para simplificar tarefas como a conexão à rede WiFi, armazenamento de dados e atualizações remotas.
\\\null \quad Esses equipamentos são capazes de medir temperatura ($^\circ$C) e umidade do ar (\%) utilizando o sensor DHT11. Este sensor foi escolhido
por sua praticidade de uso e baixo custo.
\figura[!h]{sensor.png}{Esquema de montagem para os dispositivos sensitivos}{raspsensor}{scale=0.7}
\subsection{Localização}
\quad O ambiente escolhido para a aplicação em escala reduzida é um terreno de 1000 $m²$ localizado em uma
região rural do Distrito Federal. A disposição dos equipamentos foi feita conforme a imagem \ref{planta},
há uma concentração dos sensores nos locais onde há maior fluxo de pessoas e nos lugares de interesse de
aquisição de dados.
  Este local foi escolhido pela presença constante de pessoas para a colaboração com o sistema, WiFi disponível em toda a área do terreno e pelo conhecimento prévio de valores aceitáveis de temperatura
  e umidade ao longo do ano.
\newpage
\figura{planta.jpg}{Planta baixa do ambiente escolhido para teste em escala reduzida. Os pontos em que os sensores foram instalados estão indicados em vermelho}{planta}{scale=1.4}

\subsection{Comportamento Esperado}
\quad A montagem dos equipamentos foi realizada próxima ao soltício de verão. Neste contexto, o comportamento esperado para estes sensores é o seguinte:
\begin{itemize}
  \item Período de chuvas (do processo de montagem à meados de maio):
  \begin{itemize}
    \item Valores elevados de temperatura durante o dia;
    \item Valores de temperatura amenos durante a noite;
    \item Valores de umidade mais elevados;

  \end{itemize}
  \item Período de seca (meados de maio à meados de outubro):
  \begin{itemize}
    \item Valores de temperatura elevados durante o dia;
    \item Valores de temperatura baixos durante a noite;
    \item Valores de umidade decrescendo com o passar dos dias.
  \end{itemize}
  \item Sensores externos à residência devem ter variações maiores de temperatura e umidade.
\end{itemize}


%%%%%%%%%%%%%%%%%%%%%%%%%%%%%%%%%%%%%%%%%%%%%%%%%%%%%%%%%%%%%%%%%%%%%%%%%%%%%%%%
%%%%%%%%%%%%%%%%%%%%%%%%%%%%%%%%%%%%%%%%%%%%%%%%%%%%%%%%%%%%%%%%%%%%%%%%%%%%%%%%
%%%%%%%%%%%%%%%%%%%%%%%%%%%%%%%%%%%%%%%%%%%%%%%%%%%%%%%%%%%%%%%%%%%%%%%%%%%%%%%%
