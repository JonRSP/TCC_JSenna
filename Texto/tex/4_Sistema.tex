
\quad Este capítulo trata sobre a implementação do sistema SenseHera e dos componentes necessários para seu funcionamento. A seção \ref{sec:construcao} descreve ao desenvolvimento do sistema, a seção \ref{sec:ambiente} refere-se à construção do ambiente \acrshort{IoT} em escala reduzida e a seção \ref{sec:dificuldades} trata das dificuldades encontradas durante a execução do trabalho.
\section{Construção do Sistema}
\label{sec:construcao}
\\\null \quad O propósito deste sistema é a gestão das informações produzidas pelos sensores conectados à ele, agir como intermediário entre os usuários e os dados, possibilitando a colaboração dos usuários com informações sobre fatos e sensações do ambiente e, a partir dos dados coletados pelos sensores e fornecidos pelos usuários, produzir uma noção de qualidade dos sensores.

\subsection{Requisitos}
\null \quad Os requisitos que guiaram a construção do sistema estão elencados na lista a seguir:
\begin{itemize}
  \item Funcionar em ambientes de baixa capacidade computacional;
  \item Armazenar informações coletadas por sensores;
  \item Apresentar as informações dos sensores de forma simplificada ao usuário;
  \item Escalabilidade e facilidade para adição de sensores;
  \item Permitir o envio de informações sobre fatos e sensações do ambiente pelo usuário;
  \item Baseado nos dados coletados pelos sensores e informações enviadas pelos usuários, calcular uma nota para os dispositivos sensitivos, gerando uma noção de qualidade.
\end{itemize}

\subsection{Funcionamento}
\\\null \quad As Figuras \ref{funcionamento1} e \ref{funcionamento2} mostram, de forma simplificada, o funcionamento do sistema. Na Figura \ref{funcionamento1} a interação I consiste nas seguintes fases:

\begin{itemize}
  \item O sensor envia mensagens em formato JSON dos dados coletados para o servidor para que sejam armazenados;
  \item O servidor envia o código identificador referente ao sensor ao dispositivo sensitivo.
\end{itemize}

\null Já a interação II possui as seguintes componentes:
\begin{itemize}
  \item Acesso dos usuários às informações armazenadas e/ou processadas no sistema;
  \item Disponibilização de perguntas a serem respondidas pelos usuários;
  \item Envio das percepções e sensações do usuário sobre o ambiente para o servidor.
\end{itemize}

\figura[!h]{solucao1.jpg}{Esquema de funcionamento do sistema}{funcionamento1}{scale=0.9}

\\\null \quad A Figura \ref{funcionamento2} indica a forma em que o usuário pode auxiliar com informações importantes sobre suas percepções e sobre o ambiente, fornecendo uma contextualização para o sistema, como por exemplo:
\begin{itemize}
  \item Quantidade de aberturas;
  \item Fontes de calor;
  \item Percepções de frio e calor;
  \item Noções sobre conforto.
\end{itemize}

\figura[!h]{solucao2.jpg}{Esquema de participação dos usuários no sistema}{funcionamento2}{scale=0.7}

\subsection{Servidor}
\nul \quad Foi contratado um serviço online de baixo custo (Amazon Lightsail \cite{lightsail}) para o funcionamento do sistema; a máquina virtual utilizada possui as seguintes características:
\begin{itemize}
  \item Processador de um núcleo;
  \item 512MB de memória RAM;
  \item 20 GB de SSD de armazenamento.
\end{itemize}
\\\null \quad Nessa máquina virtual foram instalados os componentes necessários para a realização do trabalho, como por exemplo:
\begin{itemize}
  \item Ubuntu 18.0;
  \item Python 2.7;
  \item Framework Django;
  \item PostgreSQL;
  \item Servidor Apache;
  \item Git.
\end{itemize}

\subsection{Banco de Dados}
\null \quad O banco de dados relacional desenvolvido segue o \acrfull{DER} da Figura \ref{DER} e foi implementado no \acrfull{SGBD} PostgreSQL por ser um software livre e com capacidade razoável para lidar com grandes volumes de dados.

\figura[!h]{tccDER.jpg}{\acrlong{DER}}{DER}{scale=0.5}
\newpage
\subsection{O Sistema de Pontuação}
\label{subsec:pontuacao}
\null \quad Para a implementação do trabalho era imperativo o desenvolvimento de um sistema de pontuação a fim de se avaliar os sensores. Para atingir este objetivo foi pensado nas seguintes formas de avaliação:
\begin{itemize}
  \item Modificar a pontuação
    \begin{itemize}
      \item A cada N leituras verifica-se a proximidade da média dessas N leituras com a média do mesmo período do dia nos M dias anteriores. A proximidade desses valores indica uma provável corretude dos dados gerados, enquanto a diferença indica possíveis erros de leitura.
    \end{itemize}
  \item Aumentar a pontuação
  \begin{itemize}
    \item A colaboração dos usuários com informações sobre suas percepções: por hora, se houverem interações, é verificado se há uma concentração nas respostas escolhidas. Por se tratarem de informações sobre sensações, não é possível afirmar que diferenças indicam erros.
  \end{itemize}
  \item Penalização
    \begin{itemize}
      \item Interrupções no envio de dados: se há uma interrupção no envio de dados, não há dados para serem considerados pelo sistema;
      \item Dados muito constantes: se nas últimas N horas não houve variação alguma entre as leituras é possível presumir que há algum erro na aquisição de dados por parte do sensor.
    \end{itemize}
\end{itemize}

\subsection{A Interface do Sistema}
\null \quad Nesta seção é exposta a interface desenvolvida e suas funcionalidades. Na Figura \ref{inicial}
é mostrada a página inicial, na qual há uma pequena explicação sobre o propósito do site e os links para os repositórios do GitHub onde se encontram disponíveis os arquivos da implementação do sistema.

\figura[!h]{inicial.png}{Página inicial do site}{inicial}{height=250,width=\textwidth}
\newpage
\null \quad Ao clicar na aba "Dados" (botão localizado na barra de navegação), o usuário é redirecionado para a página principal do site, que contém informações sobre os sensores, bem como o número de leituras armazenadas e distribuição do número de leituras por sensor, como mostra a Figura \ref{principal}.

\figura[!h]{principal.png}{Página principal do site}{principal}{height=250,width=\textwidth}
\newpage
\null \quad Ao clicar na descrição de um sensor, o sistema retorna a página de detalhes para aquele sensor, que possui informações sobre a distribuição de leituras ao longo do tempo, leituras mais recentes referentes ao sensor, gráficos a respeito das informações coletadas nas últimas 24 horas e gráficos sobre a média histórica das leituras desse sensor, como mostra a Figura \ref{pSensor}.
\newpage
\figura[!h]{pSensor.png}{Página de detalhes de um sensor}{pSensor}{height=250,width=\textwidth}


\\\null \quad Para que os usuários possam fornecer informações sobre suas sensações e percepções, é necessário que o interessado em colaborar escaneie o QR-Code localizado próximo ao sensor, o que o redirecionará para uma página com uma pergunta aleatória referente ao dispositivo, como mostra a Figura \ref{pergunta}.

\figura[!h]{pergunta.png}{Página com uma pergunta aleatória referente ao sensor}{pergunta}{height=200,width=\textwidth}

\\\null \quad Para que os usuários administradores possam cadastrar novos responsáveis, alterar informações sobre instâncias de sensores e suas categorias, adicionar novas perguntas e respostas possíveis, foi utilizada a ferramenta de adminstração fornecida pelo Framework Django, como mostra a Figura \ref{admin}.

\figura[!h]{admin.png}{Ferramenta de administração fornecida pelo Framework Django}{admin}{scale=0.5}


\section{Construção do Ambiente IoT}
\label{sec:ambiente}
\\\null \quad O ambiente \acrshort{IoT} em escala reduzida utilizado nesta seção é essencial para o teste de conceito
do sistema proposto. A alimentação de dados pelos componentes da rede é necessária para que haja
dados suficientes para o processamento do programa desenvolvido. As informações geradas por esses dispositivos, ao se comunicarem com o sistema, em colaboração
com os usuários, permitem o pleno funcionamento da proposta.

\subsection{Dispositivos Sensitivos}
\quad Foram construídos 5 dispositivos sensitivos utilizando a plataforma Raspberry Pi 0 W seguindo o esquema da Figura \ref{raspsensor}. Esta placa
foi escolhida por seu baixo valor de custo, seu desempenho computacional proporcional ao custo, sua capacidade de conexão \textit{wireless} disponível diretamente na placa, sem necessidade de equipamentos extras e a possibilidade de executar um sistema operacional baseado em linux para simplificar tarefas como a conexão à rede WiFi, armazenamento de dados e atualizações remotas.

\\\null \quad Esses equipamentos são capazes de medir temperatura ($^\circ$C) e umidade do ar (\%) utilizando o sensor DHT11. Este sensor foi escolhido
por sua praticidade de uso e baixo custo.
\\\\\\
\figura[!h]{sensor.png}{Esquema de montagem para os dispositivos sensitivos}{raspsensor}{scale=0.8}
\\\null \quad Para o envio e coleta das informações geradas pelo sensor DHT11, foram utilizadas respectivamente as bibliotecas python Requests \cite{Requests} e Adafruit DHT \cite{AdafruitDHT} em um script que segue o fluxograma da Figura \ref{fluxogramaSensor}.

\figura[!h]{fluxogramaSensor.jpg}{Fluxograma do funcionamento básico de um sensor}{fluxogramaSensor}{scale=0.57}
\newpage
\subsection{Interação Sensores-Servidor}

\\\null \quad A mensagem enviada pelos sensores é uma \textit{string} no formato JSON formada por todas as informações estritamente necessárias para o registro no banco de dados do sistema de destino.
No escopo deste trabalho são essenciais apenas as informações sobre o código identificador do equipamento, o tipo de sensor que esse dispositivo se enquadra, como por exemplo umidade e temperatura, e suas respectivas leituras. A Figura \ref{mensagemJSON} mostra o formato da informação a ser enviada.
\\\null \quad Ao receber a informação, o sistema decodifica a mensagem JSON e cria um objeto sensor no banco de dados caso o identificador recebido seja igual a 0, cria novas categorias de sensores caso as categorias recebidas não constem no banco de dados e armazena as leituras, permitindo uma rápida e simples adição de sensores ao sistema, bastando, apenas, o envio da mensagem inicial. A chave primária desse objeto sensor recém criado é enviada do servidor para o dispositivo sensitivo o qual armazena este identificador em seu banco de dados. A partir deste momento, novos envios contam com o novo valor do identificador, o que sinaliza para o sistema que deve apenas armazenar as leituras recebidas e associá-las ao sensor cuja chave primária é igual à recebida.
\\\\\\
\figura[!h]{mensagemJSON.jpg}{Formato genérico para a mensagem JSON}{mensagemJSON}{scale=0.6}

\subsection{Localização}
\quad O ambiente escolhido para a aplicação em escala reduzida é um terreno de 2500 $m^2$ localizado em uma
região rural do Distrito Federal. A disposição dos equipamentos foi feita conforme a Figura \ref{planta},
com uma concentração dos sensores nos locais onde há maior fluxo de pessoas e nos lugares de interesse de
aquisição de dados.
\\\null \quad Este local foi escolhido pela presença constante de pessoas para a colaboração com o sistema, WiFi disponível em toda a área do terreno e pelo conhecimento prévio de valores aceitáveis de temperatura
  e umidade ao longo do ano.
\newpage

\figura{planta.jpg}{Planta baixa do ambiente escolhido para teste em escala reduzida. Os pontos em que os sensores foram instalados estão indicados em vermelho}{planta}{scale=1.4}
\newpage
Para que os usuários possam contribuir com as informações sobre suas sensações e percepções, foram colocados QR-Codes próximos aos sensores, como mostrado na Figura \ref{sensormsg}, para que sejam escaneados e então redirecionem o usuário interessado em colaborar para uma página com uma pergunta aleatória referente ao sensor mais próximo ao QR-Code escaneado.

\figura{sensormsg.jpg}{Sensor e QR-Code}{sensormsg}{scale=0.1}

\subsection{Comportamento Esperado}
\quad A montagem dos equipamentos foi realizada próxima ao solstício de verão. Neste contexto, o comportamento esperado para estes sensores é o seguinte:
\begin{itemize}
  \item Período chuvoso (do processo de montagem (dezembro) à meados de maio):
  \begin{itemize}
    \item Valores elevados de temperatura durante o dia;
    \item Valores de temperatura amenos durante a noite;
    \item Valores de umidade mais elevados.

  \end{itemize}
  \item Período de seca (meados de maio à meados de outubro):
  \begin{itemize}
    \item Valores de temperatura elevados durante o dia;
    \item Valores de temperatura baixos durante a noite;
    \item Valores de umidade decrescendo com o passar dos dias.
  \end{itemize}
  \item Sensores externos à residência devem ter variações maiores de temperatura e umidade.
\end{itemize}

\section{Dificuldades Encontradas}
\label{sec:dificuldades}
\quad Durante a execução do projeto, muitas dificuldades foram superadas para que os objetivos traçados fossem atingidos.
\subsection{Localização dos Sensores}
\quad A montagem inicial dos sensores foi realizada em uma região central da cidade de Brasília. Entretanto, alguns meses após o começo da coleta de dados foi necessária uma mudança para a localização descrita nos procedimentos do trabalho.
\\\null \quad Foi inevitável a eliminação dos dados produzidos até a data da transferência de localidade por se tratarem de regiões distintas, o que traria inconsistências ao sistema. Devido ao processo de mudança, também houve um grande atraso no recomeço da aquisição de dados, reduzindo consideravelmente a quantidade de registros que seriam coletados até o final da realização do trabalho.

\subsection{Servidor}
\quad Inicialmente, o servidor seria localizado em uma rede local e implementado em uma placa de desenvolvimento Raspberry Pi modelo B+ que possui as seguintes características físicas:
\begin{itemize}
  \item Tamanho: 85mm x 56mm;
  \item Processador ARM em clock de 700MHz;
  \item 512MB de memória RAM;
  \item 40 pinos GPIO;
  \item 4 portas USB 2.0.
\end{itemize}
\null\quad Anteriormente à mudança de localidade, esta placa funcionou de forma adequada, armazenando os dados enquanto os outros elementos do sistema eram desenvolvidos. Após a mudança de localidade, devido a diversas falhas na rede elétrica, houve uma corrupção dos dados armazenados na placa, o que levou à perda dos dados coletados até então, por volta de 300.000 (trezentos mil) registros de leituras. Após essa perda de dados, foi contratado um serviço externo e online, o que acarretou em um novo atraso no início da coleta de dados devido à necessidade da configuração referentes a um servidor remoto e online.

\subsection{\acrlong{SGBD}}
\quad Inicialmente, o \acrshort{SGBD} escolhido foi o MySQL. Entretanto, seguindo a implementação do \acrshort{DER} desenvolvido para este trabalho, este \acrlong{SGBD} não possuia um desempenho satisfatório ao atingir mais de 900.000 (novecentos mil) registros de leituras, limitando consideravelmente a capacidade do servidor.
\\\null \quad Devido à quantidade de dados já produzidos, à baixa capacidade computacional do servidor e  inconsistências no formato do arquivo de exportação do \acrshort{SGBD} previamente utilizado, não foi possível a importação no PostgreSQL das informações contidas no MySQL, levando à perda de uma considerável amostra de dados.
