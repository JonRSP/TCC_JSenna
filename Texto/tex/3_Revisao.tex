\newcommand{\texCommand}[1]{\texttt{\textbackslash{#1}}}%

\newcommand{\exemplo}[1]{%
\vspace{\baselineskip}%
\noindent\fbox{\begin{minipage}{\textwidth}#1\end{minipage}}%
\\\vspace{\baselineskip}}%

\newcommand{\exemploVerbatim}[1]{%
\vspace{\baselineskip}%
\noindent\fbox{\begin{minipage}{\textwidth}%
#1\end{minipage}}%
\\\vspace{\baselineskip}}%


\quad Este capítulo descreve os métodos utilizados para a aquisição da bibliografia utilizada. Foi definido um processo baseado no método de revisão sistemática definido por Barbara Kitchenham e Stuart Charters em 2007 \cite{rsguidelines}, apenas para que fosse seguido um protocolo para a aquisição de referências bibliográficas. Esta revisão consiste em três fases:
\begin{itemize}
  \item Planejar;
  \item Executar;
  \item Documentar.
\end{itemize}
\quad A fase de planejamento constitui-se na especificação do protocolo.
A fase de execução representa a coleta dos dados de forma a atender as especificações
exigidas na fase de planejamento. A etapa de documentação foi a própria escrita deste capítulo.
% A fase de documentação implica na consolidação dos dados obtidos.

\section{Planejamento}
\quad O objetivo desta revisão sistemática é a identificação de trabalhos acadêmicos
que expõem resultados, projeções, explicações ou elucidações sobre o tema de armazenamento ou gerenciamento
de metadados para \acrlong{IoT} de forma colaborativa ou não, com o propósito de que haja uma metodologia durante a construção das referêcias do presente trabalho.

\subsection{Questões de Estudo}
\quad Esta revisão tem como objetivo responder às seguintes questões:
\begin{itemize}
  \item Há estudos sobre o armazenamento e gerenciamento de metadados para \acrlong{IoT}?
  \item Quais são os métodos mais utilizados para o armazenamento e gerenciamento de metadados para \acrshort{IoT}?
  \item Há estudos sobre a utilização de abordagens colaborativas em \acrlong{IoT}?
  \item Em quais aplicações os metadados estão sendo utilizados nos trabalhos?
\end{itemize}

\subsection{Estratégia de Busca}
\quad Foi determinada uma estratégia para a realização das buscas nas bases de dados escolhidas conforme a seguir:

\paragraph{Definição da \textit{String} de Busca}
\begin{itemize}
  \item \textbf{População}: Utilizou-se como tema principal metadados para \acrshort{IoT}. Para procurar, foram utilizadas as palavas-chave 'Internet of things metadata', 'IoT metadata' e 'Collaborative IoT';
  \item \textbf{Intervenção}: O foco é verificar \textit{middlewares}, modelos e esquemas. Os termos utilizados para pesquisa foram 'middleware', 'management', 'model' e 'schema';
  \item \textbf{Comparação}: O foco deste trabalho não se limitou a estudos comparativos;
  \item \textbf{Resultado}: Tem-se como objetivo a procura de avaliações, definições, validações e implementações de middlewares e/ou esquemas utilizados em pesquisas científicas. Desta forma, foram obtidas as seguintes palavras-chave
  'validation', 'evaluation' e 'implementation'.
\end{itemize}

\quad A \textit{string} de busca gerada é a seguinte:
(('Internet of things metadata') \textbf{OR} ('IoT metadata') \textbf{OR} ('Collaborative IoT')) \textbf{AND} (('middleware') \textbf{OR} ('schema') \textbf{OR} ('management') \textbf{OR} ('model')) \textbf{AND} (('validation') \textbf{OR} ('evaluation') \textbf{OR} ('implementation'))

\paragraph{Fontes de Busca}
\subparagraph{}
 Foram escolhidas as seguintes bases digitais para que as buscas fossem realizadas:
\begin{itemize}
  \item Google Acadêmico (https://scholar.google.com.br/)
  \item JSTOR (https://www.jstor.org/)
  \item Periódicos CAPES (https://www.periodicos.capes.gov.br/)
\end{itemize}
\quad Bases escolhidas devido a sua relevância e sua grande abrangência sobre diversos temas.

\paragraph{Idioma}
\subparagraph{}
\quad O idioma de preferência para seleção de artigos foi a língua inglesa, entretanto, trabalhos científicos escritos em
língua portuguesa não foram descartados, desde que atingissem os requisitos para inclusão.

\subsubsection{Seleção dos Estudos}
\paragraph{Critérios de Inclusão e Exclusão}
\subparagraph{Critérios de Inclusão}
\begin{itemize}
  \item O texto integral dos trabalhos devem estar disponíveis nas bases de dados escolhidas previamente;
  \item Serão consideradas apenas as publicações posteriores à 2008 (dez anos antes do início do projeto), salvo em casos de fontes relevantes que contenham definições necessárias para a realização deste trabalho;
  \item O trabalho deve fazer menção a "metadados em ambientes \acrlong{IoT}".
\end{itemize}

\subparagraph{Critérios de Exclusão}
\begin{itemize}
  \item Trabalhos publicados anteriormente à 2008, exceto fontes relevantes que contenham definições pertinentes para este estudo;
  \item Trabalhos parcialmente disponíveis nas bases de dados digitais escolhidas;
  \item Publicações cujo texto não trate do tema "metadados em ambientes \acrshort{IoT}";
  \item Trabalhos que não contenham propostas, comparações ou avaliações de métodos para o gerênciamento ou armazenamento de metadados;
  \item Trabalhos publicados em mútiplas bases de dados. O trabalho será contado apenas uma vez.
\end{itemize}

\section{Execução}
\subsection{Processo de Seleção dos Estudos}
\quad Os artigos obtidos por meio da estratégia descrita passaram por um processo de avaliação metódica, com base nos critérios anteriormente especificados.
Desta forma, os artigos que atingiram os parâmetros estabelecidos foram adicionados à base de estudos da revisão sistemática. A estratégia para a pesquisa e seleção foi:
\begin{enumerate}
  \item Pesquisa de trabalhos científicos nas bases de dados definidas utilizando as \textit{strings} de busca;
  \item Leitura do título, resumo, palavras chave e data de publicação, aplicando os critérios de inclusão e exclusão definidos;
  \item Leitura da introdução e conclusão dos trabalhos que foram mantidos na fase anterior;
  \item Os trabalhos resultantes serão lidos por completo, e as informações pertinentes serão coletadas.
\end{enumerate}

\subsection{Resultado da Seleção de Estudos}
\paragraph{Fase I - Total de estudos obtidos}
\subparagraph{}
\quad Utilizando todas as combinações possíveis da \textit{string} de busca foram encontrados as seguintes quantidades de estudos sobre o tema:
\begin{itemize}
  \item Google Acadêmico - 40 estudos;
  \item JSTOR - 3 estudos;
  \item Periódicos CAPES - 25 estudos.
\end{itemize}
\paragraph{Fase II - Estudos que possuem relação com o tema}
\subparagraph{}
\quad Nesta fase realizou-se a leitura do título, resumo, palavras chave e identificação da data de publicação.
Aplicados os critérios estabelecidos, foi obtido o seguinte resultado:
\begin{itemize}
  \item Google Acadêmico - 24 estudos;
  \item JSTOR - 3 estudos;
  \item Periódicos CAPES - 13 estudos.
\end{itemize}
\paragraph{Fase III - Estudos que possuem relação com o tema}
\subparagraph{}
\quad Os artigos coletados passaram pela leitura da introdução e conclusão.
Aplicados os critérios dessa fase, o seguinte resultado foi obtido:
\begin{itemize}
  \item Google Acadêmico - 12 estudos;
  \item JSTOR - 1 estudo;
  \item Periódicos CAPES - 7 estudos.
\end{itemize}
\paragraph{Fase IV - Leitura dos Estudos}
\subparagraph{}
\quad O material que passou pela seleção das fases II e III foi lido integralmente e as
informações pertinentes foram incluídas nas seções adequadas.
\subsubsection{Respostas às Questões de Estudo}
\begin{itemize}
  \item Há estudos sobre o armazenamento de metadados para \acrlong{IoT}?
 \begin{itemize}
    \item \textit{Existem estudos sobre o armazenamento de metadados para \acrshort{IoT}; entretanto, 4 deles tratam
    sobre este tema específico, representando um total de, aproximadamente, 20\% dos artigos considerados úteis para o trabalho} \cite{armazenamento1} \cite{armazenamento2} \cite{armazenamento3} \cite{armazenamento4}.
  \end{itemize}
  \item Quais são os métodos mais utilizados para o armazenamento de metadados para \acrshort{IoT}?
    \begin{itemize}
    \item \textit{Foi descoberto que não há um consenso sobre o método de armazenamento de metadados \acrshort{IoT}. Cada estudo introduz um método diferente} \cite{armazenamento1} \cite{armazenamento2} \cite{armazenamento3} \cite{armazenamento4}.
  \end{itemize}
  \item Há estudos sobre a utilização de abordagens colaborativas em \acrlong{IoT}?
  \begin{itemize}
    \item \textit{Há apenas um estudo que utiliza abordagens colaborativas para a obtenção de metadados para \acrshort{IoT}} \cite{collaborative}.
  \end{itemize}
  \item Em quais aplicações os metadados estão sendo utilizados nos trabalhos?
  \begin{itemize}
    \item \textit{As smart buildings (construções inteligentes) foram as principais aplicações em que os metadados são utilizados em \acrshort{IoT}} \cite{collaborative} \cite{buildings1}.
  \end{itemize}
\end{itemize}

\section{Conclusão da Revisão da Literatura}
\quad Após a utilização da \textit{string} de busca e das fases II e III, foi obtido um total de 20 estudos significativos,
ou seja, 29\% do total de estudos coletados.




%%%%%%%%%%%%%%%%%%%%%%%%%%%%%%%%%%%%%%%%%%%%%%%%%%%%%%%%%%%%%%%%%%%%%%%%%%%%%%%%
%%%%%%%%%%%%%%%%%%%%%%%%%%%%%%%%%%%%%%%%%%%%%%%%%%%%%%%%%%%%%%%%%%%%%%%%%%%%%%%%
%%%%%%%%%%%%%%%%%%%%%%%%%%%%%%%%%%%%%%%%%%%%%%%%%%%%%%%%%%%%%%%%%%%%%%%%%%%%%%%%
