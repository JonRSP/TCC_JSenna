O presente trabalho visa a obtenção e gerenciamento de metadados em informações provenientes de
dispositivos categorizados em \acrfull{IoT}. %expandiiiiiiiiir


\section{Internet das Coisas}%
No cenário contemporâneo, um assunto que tem sido bastante abordado é a
\acrlong{IoT}, que, como o nome sugere, seria uma “internet em todas as
coisas”. A proposta de \acrshort{IoT} consiste em vários objetos do cotidiano trocando informações
mutuamente, através da internet, para serem mais eficientes e realizarem diversas tarefas.
Os objetos passam a agir de forma mais inteligente e sensorial, de modo a favorecer diversos setores como:
indústria, hospitais, agropecuária, transporte público e muitos outros. A partir desta
disponibilidade astronômica de recursos, a \acrlong{IoT} está a se tornar uma ferramenta
de grande importância para a humanidade.\\\\
Um dos objetivos principais da \acrlong{IoT} é permitir
que humanos e máquinas a ter maior consciência de seus arredores.
 Esse maior entendimento do seu ambiente é possível através da utilização
 de diversos tipos dispositivos sensitivos (sensores) e após a percepção
 de seu ambiente é possível realizar ações (atuadores) ou análises.
\subsection{Definição}
	Em 2012, a \acrfull{ITU} realizou estudos sobre infraestrutura
	de informação global, aspectos de procolos de internet e redes da próxima geração.
	A partir desse estudo foi construída a recomendação ITU-T Y.2060 \cite{ITU} que trata sobre a \acrlong{IoT}
	e possui o intuito de esclarecer o conceito e o escopo de \acrshort{IoT}, identificar
	as características fundamentais e os requerimentos de alto-nivel.
\\	No documento produzido pela \acrshort{ITU}, foram consolidadadas as definições de:
	\begin{itemize}
		\item \acrlong{IoT}, "uma infraestrutura global para a Sociedade de Informações, permitindo serviços avançados ao
		interconectar (fisicamente e virtualmente) coisas devido à existência e evolução da interoperabilidade
	de tecnologias de comunicação e informação";
		\item Dispositivo, no contexto de \acrshort{IoT}, é um equipamento que, obrigatoriamente, possui a capacidade
		de comunicação e, opcionalmente, possui capacidade de sensitividade, atuação, captura de dados,
		armazenamento de dados e/ou processamento de dados;
		\item Coisas, no contexto de \acrshort{IoT}, são "objetos
	no mundo físico (objetos físicos) ou no mundo das informações (objetos virtuais), os quais são capazes de
	de serem identificados e integrados a uma rede de comunicações".
	\end{itemize}
% [1] ITU

\subsection{Desafios}
	A \acrlong{IoT} possui diversos desafios devido à sua própria concepção. Limitantes
	como infraesturura de rede, segurança, espaço de armazenamento e consumo de energia são apenas
	alguns exemplos de dificuldades a serem ultrapassadas para que a \acrshort{IoT} possa ser amplamente
	e devidamente implantada.


%%%%%%%%%%%%%%%%%%%%%%%%%%%%%%%%%%%%%%%%%%%%%%%%%%%%%%%%%%%%%%%%%%%%%%%%%%%%%%%%
%%%%%%%%%%%%%%%%%%%%%%%%%%%%%%%%%%%%%%%%%%%%%%%%%%%%%%%%%%%%%%%%%%%%%%%%%%%%%%%%
%%%%%%%%%%%%%%%%%%%%%%%%%%%%%%%%%%%%%%%%%%%%%%%%%%%%%%%%%%%%%%%%%%%%%%%%%%%%%%%%
\section{dados}%
	\begin{itemize}
		\item definição
		\item valores e estatísticas
		\item desafios
	\end{itemize}
	\subsection{metadados}
		\begin{itemize}
			\item definição
			\item usabilidade
			\item desafios
		\end{itemize}
	\subsection{Taxonomia}
		\begin{itemize}
			\item definição
			\item usabilidade
			\item desafios
		\end{itemize}
	\subsection{Ontologia}
		\begin{itemize}
			\item definição
		 	\item usabilidade
			\item desafios
		\end{itemize}
