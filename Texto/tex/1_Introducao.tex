O presente trabalho visa a obtenção e gerenciamento de metadados em informações provenientes de
dispositivos categorizados em \acrfull{IoT}. %expandiiiiiiiiir


\section{Internet das Coisas}%
A \acrlong{IoT} é um novo paradigma tecnológico idealizado como uma conexão global
de máquinas e dispositivos capazes de interagir entre si. A proposta de \acrshort{IoT} consiste em vários objetos do cotidiano trocando informações
mutuamente, através da internet, para serem mais eficientes e realizarem diversas tarefas.
Os objetos passam a agir de forma mais inteligente e sensorial, de modo a favorecer diversos setores como:
indústria, hospitais, agropecuária, transporte público e muitos outros. A partir desta
disponibilidade astronômica de recursos, a \acrshort{IoT} é reconhecida com uma das áreas mais importantes
em termos de tecnologia do futuro e está recebendo cada vez mais atenção de desenvolvedores, usuários e indústrias.
\\\\Um dos objetivos principais da \acrlong{IoT} é permitir
que humanos e máquinas possuam maior consciência de seus arredores.
 Esse maior entendimento do seu ambiente é possível através da utilização
 de diversos tipos dispositivos sensitivos (sensores) e após a percepção
 de seu ambiente é possível realizar ações (atuadores) ou análises.\\\\
 A \acrlong{IoT} surgiu a partir do cojunto de diferentes visões como podemos observar na \refFig{refiotvision}, cada qual com seus objetivos específicos.
 \figura[!h]{visionsiot}{O paradigma \acrlong{IoT} como um resultado de diferentes visões}{refiotvision}{}%
\subsection{Definição}
	Em 2012, a \acrfull{ITU} realizou estudos sobre infraestrutura
	de informação global, aspectos de procolos de internet e redes da próxima geração.
	A partir desse estudo foi construída a recomendação ITU-T Y.2060 \cite{ITU} que trata sobre a \acrlong{IoT}
	e possui o intuito de esclarecer o conceito e o escopo de \acrshort{IoT}, identificar
	as características fundamentais e os requerimentos de alto-nivel.
\\	No documento produzido pela \acrshort{ITU}, foram consolidadadas as definições de:
	\begin{itemize}
		\item \acrlong{IoT}, "uma infraestrutura global para a Sociedade de Informações, permitindo serviços avançados ao
		interconectar (fisicamente e virtualmente) coisas devido à existência e evolução da interoperabilidade
	de tecnologias de comunicação e informação" \cite{ITU};
		\item Dispositivo, no contexto de \acrshort{IoT}, é um equipamento que, obrigatoriamente, possui a capacidade
		de comunicação e, opcionalmente, possui capacidade de sensitividade, atuação, captura de dados,
		armazenamento de dados e/ou processamento de dados \cite{ITU};
		\item Coisas, no contexto de \acrshort{IoT}, são "objetos
	no mundo físico (objetos físicos) ou no mundo das informações (objetos virtuais), os quais são capazes de
	de serem identificados e integrados a uma rede de comunicações". Objetos físicos podem sentir, atuar e conectar.
	Obejtos virtuais podem ser armazenados, processados e acessados.\cite{ITU}
	\end{itemize}
	\figura[!h]{iottec}{Visão geral técnica de \acrshort{IoT}}{refiottec}{}%
% [1] ITU\\
\subsection{Tecnologias excenciais}
	\subsubsection{\acrfull{RFID}}
		\begin{itemize}
			\item Esta tecnologia permite identificação automática e captura de informação por meio de rádio frequência.
			Divide-se dispositivos \acrshort{RFID} em duas grandes categorias, ativos e passivos. Dispositivos ativos dependem
			de uma fonte de energia constante para manter ativa e transmitir a informação. Dispositivos passivos não necessitam de energia constante,
			um campo eletromagnético energiza o dispositivo, o qual se torna apto a transferir a informação contida nele.
			\cite{refrfid}
		\end{itemize}
	\subsubsection{\acrfull{RSSF}}
		\begin{itemize}
			\item Esta tecnologia consiste na distribuição de dispositivos sensitivos autônomos para monitorar condições físicas ou
			ambientais e podem cooperar com sistemas \acrshort{RFID} para medir de forma mais eficáz localização, temperatura e movimentação, por exemplo.
			\cite{IOTS}
		\end{itemize}
	\subsubsection{Middleware}
		\begin{itemize}
			\item O middleware é a camada de abstração entre aplicações de software para tornar mais fácil para os desenvolvedores
			 realizar a comunicação entre softwares e operações de recebimento e envio de dados. O objetivo do middleware no contexto
			  de \acrshort{IoT} é simplificar a integração entre dispositivos heterogêneos.
		\end{itemize}
	\subsubsection{Computação em núvem}
		\begin{itemize}
			\item Computação em núvem é um modelo para acesso de recursos compartilhados conforme a necessidade de um serviço. Um dos resultados mais notáveis
			da \acrshort{IoT} é a enorme quantidade de dados gerados por dispositivos conectados à internet \cite{IOTV}. A computação em núvem é importante para o contexto de \acrlong{IoT}
			ao permitir um ambiente com alta escalabilidade.
		\end{itemize}
	\subsubsection{Aplicações de software}
		\begin{itemize}
			\item Aplicações \acrshort{IoT} permitem interações dispositivo-dispositivo e humano-dispositivo de uma forma confiável e robusta.
			AS aplicações nos dispositivos devem garantir que as informações são recebidas e processadas de maneira adequada, no momento
			adequado.

		\end{itemize}

\end{itemize}

\subsection{Desafios}
	A \acrlong{IoT} possui diversos desafios devido à sua própria concepção, essas dificuldades devem ser ultrapassadas para que a
	\acrshort{IoT} possa ser amplamente e devidamente implantada. Alguns fatores críticos podem ser elencados:
	\subsubsection{Infraesturura de rede}
		O custo para interconectar os dispositivos é alto. Para uma grande rede de sensores é necessário a distribuição de toda infraestrutua,
		de cabeamento ou infraestrutura sem fio.
	\subsubsection{Segurança}
		Uma das principais dificuldades num ambiente de \acrlong{IoT} é a segurança dos dados que são coletados pela grande quantidade de
		dispositivos sensitivos da rede. Uma invasão de um sistema crítico pode significar consequências graves em um ambiente automatizado.
  \subsubsection{Espaço de armazenamento}
	  Um ambiente \acrshort{IoT} gera uma grande quantidade de dados, suponhamos um sistema de uma cidade inteligente que possui 10000 sensores
 	  de diversas categorias, cada sensor gerando uma mensagem de 10 kB a cada 5 minutos, por dia, temos um total de aproximadamente 29 GB de dados
	  após um ano teríamos 10,5 TB de dados apenas para esta cidade.
  \subsubsection{Consumo de energia}
		É imperativo que o consumo de energia dos dispositivos sensitivos seja o menor possível, em muitos cenários é improvável a presença de uma
		rede de energia elétrica e até mesmo de manutenção constante, então os equipamentos devem conseguir se manter funcionais por meio de baterias por
		uma quantidade de tempo considerável.


%%%%%%%%%%%%%%%%%%%%%%%%%%%%%%%%%%%%%%%%%%%%%%%%%%%%%%%%%%%%%%%%%%%%%%%%%%%%%%%%
%%%%%%%%%%%%%%%%%%%%%%%%%%%%%%%%%%%%%%%%%%%%%%%%%%%%%%%%%%%%%%%%%%%%%%%%%%%%%%%%
%%%%%%%%%%%%%%%%%%%%%%%%%%%%%%%%%%%%%%%%%%%%%%%%%%%%%%%%%%%%%%%%%%%%%%%%%%%%%%%%
\section{dados}%
	Considerando o ambiente de \acrshort{IoT}, a quantidade de dados gerados por unidade de tempo é gigantesca,

	\subsection{metadados}
		\begin{itemize}
			\item definição
			\item usabilidade
			\item desafios
		\end{itemize}
	\subsection{Taxonomia}
		\begin{itemize}
			\item definição
			\item usabilidade
			\item desafios
		\end{itemize}
	\subsection{Ontologia}
		\begin{itemize}
			\item definição
		 	\item usabilidade
			\item desafios
		\end{itemize}
