\section{Contextualização}
\qquad
O avanço de inúmeras tecnologias incluindo sensores, atuadores, computação em nuvem e o despontamento de
incontáveis equipamentos com capacidade de conexão com a Internet tem colaborado com a necessidade humana pela interoperabilidade entre diversas classes de dispositivos.
\\ \null
\qquad
A \acrfull{IoT}, também chamada de Internet de Tudo ou até mesmo Internet Industrial, tem como objetivo primário % fazer citação
permitir que humanos e máquinas compreendam melhor o ambiente que os envolve, usando as informações geradas por
diversos dispositivos sensitivos. \acrshort{IoT} é um novo paradigma tecnológico planejado para ser uma rede %segundo fulano, (data,p.pagina)
global de máquinas e dispositivos capazes de interagir entre si e com o ambiente ao seu redor.
A \acrshort{IoT} é reconhecida como uma das áreas mais importantes da tecnologia do futuro pelo fato de %por 1, 2 ênfase na interoperabilidade e 3 implementação...
poder ser implementada em diversos casos de uso e enfatizar a interoperabilidade entre objetos e pessoas.
\\ \null
\qquad A estimativa do Conselho Nacional de Inteligência norte americano é de que, até 2025, objetos do cotidiano como embalagens de alimentos, mobília e documentos
poderão estar conectados à Internet \cite{intelsix}. Considerando-se que haverá uma grande variedade de tipos de equipamentos com a necessidade de interoperabilidade mencionada anteriormente, %qualificar a conexão?
é possível afirmar que uma enorme quantidade destes estarão interconectados. Neste contexto, %trocar equipamentos
é esperado que a quantidade de dados seja superior ao número de dispositivos, em uma ordem de grandeza ainda maior.
Estes dados, se utilizados de forma correta, considerando os contextos específicos e o nível de qualidade dessas informações, poderão
contribuir muito para avanços em pesquisas e na melhoria da qualidade de serviços.
\section{Objetivos}
\subsection{Objetivo Geral}
\qquad O objetivo geral do presente trabalho é desenvolver um sistema colaborativo que gerencie e avalie a qualidade de dados
gerados em um ambiente de \acrlong{IoT}.
\subsection{Objetivos Específicos}
Para a realização do objetivo geral, os seguintes objetivos específicos devem ser atingidos:
\begin{itemize}
  \item Desenvolver um ambiente \acrshort{IoT} em escala reduzida para teste de conceito;
  \item Desenvolver um sistema web para aquisição, armazenamento e acesso das informações geradas pelos sensores;
  \item Desenvolver um módulo que realize a avaliação de qualidade de sensores e dados, a partir dos dados fornecidos pelos colaboradores.
\end{itemize}
\section{Estrutura do trabalho}
O presente trabalho é composto pelos seguintes capítulos:
\begin{itemize}
  \item \textbf{Capítulo 2 - Fundamentação Teórica:} apresentação dos conceitos necessários para a compreensão do
  trabalho. Neste capítulo serão abordados os seguintes temas:
  \begin{itemize}
    \item Sistemas colaborativos;
    \item \acrlong{IoT};
    \item Dados;
    \item Plataforma.
  \end{itemize}
  \item \textbf{Capítulo 3 - Metodologia:}
\begin{itemize}
  \item Detalhamento dos métodos utilizados para a realização do trabalho;
  \item Apresentação de um sistema similar que serviu como inspiração para o presente trabalho;
  \item Explanação do processo de levantamento de requisitos;
  \item Definição do sistema proposto e de sua arquitetura, bem como a descrição do processo de desenvolvimento.
\end{itemize}
  \item \textbf{Capítulo 4 - Prova de Conceito:} exposição dos testes planejados e realizados no sistema e seus resultados, bem como a indicação de possíveis problemas e melhorias a serem implementadas.
  \item \textbf{Capítulo 5 - Conclusão:} apresentação das conclusões advindas dos resultados obtidos, limitações e
  sugestões para trabalhos futuros.
\end{itemize}
