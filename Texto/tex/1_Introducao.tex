O presente trabalho visa a obtenção e gerenciamento de metadados em informações provenientes de
dispositivos categorizados em Internet das Coisas, Internet of Things (IoT). %expandiiiiiiiiir


\section{Internet das Coisas}%
No cenário contemporâneo, um assunto que tem sido bastante abordado é a
internet das coisas, que, como o nome sugere, seria uma “internet em todas as
coisas”. A proposta é que vários objetos do nosso dia a dia possam trocar informações
mutuamente, através da internet, para serem mais eficientes. Os objetos passam a agir
de forma mais inteligente e sensorial, de modo a favorecer diversos setores como:
indústria, hospitais, agropecuária, transporte público e muitos outros. A partir desta
disponibilidade infinita de recursos, a internet das coisas está a se tornar uma ferramenta
de grande importância.
\subsection{Definição}
	Em 2012, a União Internacional de Telecomunicações (ITU) realizou estudos sobre infraestrutura
	de informação global, aspectos de procolos de internet e redes da próxima geração.
	A partir desse estudo foi construída a recomendação ITU-T Y.2060 que trata sobre a Internet
	das Coisas e possui o intuito de esclarecer o conceito e o escopo de IoT, identificar
	as características fundamentais e os requerimentos de alto-nivel.
	No documento produzido pela ITU, foram consolidadadas as definições de:
	\begin{itemize}
		\item Internet das Coisas, "uma infraestrutura global para a Sociedade de Informações, permitindo serviços avançados ao
		interconectar (fisicamente e virtualmente) coisas devido à existência e evolução da interoperabilidade
	de tecnologias de comunicação e informação";
		\item Dispositivo, no contexto de IoT, é um equipamento que, obrigatoriamente, possui a capacidade
		de comunicação e, opcionalmente, possui capacidade de sensitividade, atuação, captura de dados,
		armazenamento de dados e/ou processamento de dados;
		\item Coisas, no contexto de IoT, são "objetos
	no mundo físico (objetos físicos) ou no mundo das informações (objetos virtuais), os quais são capazes de
	de serem identificados e integrados a uma rede de comunicações".
	\end{itemize}
% [1] ITU
\subsection{decidir nome da subseção}



\subsection{Desafios}
	asdmaodmasdmaosmdamds.


%%%%%%%%%%%%%%%%%%%%%%%%%%%%%%%%%%%%%%%%%%%%%%%%%%%%%%%%%%%%%%%%%%%%%%%%%%%%%%%%
%%%%%%%%%%%%%%%%%%%%%%%%%%%%%%%%%%%%%%%%%%%%%%%%%%%%%%%%%%%%%%%%%%%%%%%%%%%%%%%%
%%%%%%%%%%%%%%%%%%%%%%%%%%%%%%%%%%%%%%%%%%%%%%%%%%%%%%%%%%%%%%%%%%%%%%%%%%%%%%%%
\section{dados}%
	\begin{itemize}
		\item definição
		\item valores e estatísticas
		\item desafios
	\end{itemize}
	\subsection{metadados}
		\begin{itemize}
			\item definição
			\item usabilidade
			\item desafios
		\end{itemize}
	\subsection{Taxonomia}
		\begin{itemize}
			\item definição
			\item usabilidade
			\item desafios
		\end{itemize}
	\subsection{Ontologia}
		\begin{itemize}
			\item definição
		 	\item usabilidade
			\item desafios
		\end{itemize}
