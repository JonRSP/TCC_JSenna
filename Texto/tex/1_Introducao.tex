\section{Contextualização}
\qquad
O avanço de inúmeras tecnologias incluindo sensores, atuadores, computação em nuvem e o despontamento de equipamentos com capacidade de conexão com a Internet tem colaborado com a necessidade pela interoperabilidade entre diversas classes de dispositivos.
\\ \null
\qquad
A \acrlong{IoT}, do inglês \textit{Internet of things} (\acrshort{IoT}), consiste na interconexão de dispositivos sensitivos e atuadores com a finalidade de atingir um objetivo em comum \cite{giusto} e tem como objetivo primário % fazer citação
permitir que humanos e máquinas compreendam melhor o ambiente que os envolve, usando as informações geradas por
diversos dispositivos sensitivos, modificando a forma com que os usuários lidam com as tarefas do cotidiano \cite{IOTS}. \acrshort{IoT} é um novo paradigma tecnológico planejado para ser uma rede
global de máquinas e dispositivos capazes de interagir entre si e com o ambiente ao seu redor.
A \acrshort{IoT} é reconhecida como uma das áreas mais importantes da tecnologia do futuro por enfatizar a interoperabilidade entre objetos e pessoas e pelo fato de
poder ser implementada em diversos casos de uso, como por exemplo, automação predial, controle de processos produtivos e transporte inteligente \cite{IOTV}.
\\ \null
\qquad A estimativa do Conselho Nacional de Inteligência norte americano é de que, até 2025, objetos do cotidiano como embalagens de alimentos, mobília e documentos
poderão estar conectados à Internet \cite{intelsix}. Considerando-se que haverá uma grande variedade de tipos de equipamentos com a necessidade de interoperabilidade mencionada anteriormente,
é possível afirmar que uma enorme quantidade destes estarão interconectados. Neste contexto,
é esperado que a quantidade de dados seja superior ao número de dispositivos, em uma ordem de grandeza ainda maior.
\\ \null
\qquad Estes dados, se utilizados de forma correta, considerando os contextos específicos e o nível de qualidade dessas informações, poderão contribuir muito para avanços em pesquisas e na melhoria da qualidade de serviços. Para que tal objetivo seja atingido, alguns autores propuseram que os usuários, de forma colaborativa, possam contribuir com informações de metadados para os sistema de maneira a reduzir o problema da dinamicidade relacionada a \acrshort{IoT}, uma abordagem não muito utilizada no meio acadêmico \cite{collaborative}.
\section{Objetivos}
\subsection{Objetivo Geral}
\qquad O objetivo geral do presente trabalho é desenvolver um sistema colaborativo que armazene, gerencie e avalie a qualidade de dados
gerados em um ambiente de \acrlong{IoT}.
\subsection{Objetivos Específicos}
Para a realização do objetivo geral, os seguintes objetivos específicos devem ser atingidos:
\begin{itemize}
  \item Desenvolver um ambiente \acrshort{IoT} em escala reduzida para teste de conceito;
  \item Desenvolver um sistema web para aquisição, armazenamento e acesso das informações geradas pelos sensores;
  \item Desenvolver um módulo que realize a avaliação de qualidade de sensores e dados, a partir dos dados fornecidos pelos colaboradores.
\end{itemize}
\section{Estrutura do Trabalho}
O presente trabalho é composto pelos seguintes capítulos:
\begin{itemize}
  \item \textbf{Capítulo 2 - Fundamentação Teórica:} apresentação dos conceitos necessários para a compreensão do
  trabalho.
  % \begin{itemize}
  %   \item Sistemas colaborativos;
  %   \item \acrlong{IoT};
  %   \item Dados;
  %   \item Plataforma.
  % \end{itemize}
  \item \textbf{Capítulo 3 - Revisão Bibliográfica:} apresentação da metodologia utilizada para obtenção da literatura que orientou o trabalho.
  \item \textbf{Capítulo 4 - Sistema SenseHera:} apresentação do funcionamento do sistema construído, bem como seus requisitos e características.
% \begin{itemize}
%   \item Detalhamento dos métodos utilizados para a realização do trabalho;
%   \item Apresentação de um sistema similar que serviu como inspiração para o presente trabalho;
%   \item Explanação dos requisitos do sistema;
  % \item Definição do sistema proposto e de sua arquitetura, bem como a descrição do processo de desenvolvimento.
% \end{itemize}
  \item \textbf{Capítulo 5 - Resultados:} comparação da métrica utilizada pelo sistema proposto com a métrica \textit{\acrlong{OEE}} (\acrshort{OEE}), amplamente utilizada na indústria para verificação de qualidade de equipamentos.
  \item \textbf{Capítulo 6 - Conclusão:} apresentação das conclusões advindas dos resultados obtidos e
  sugestões para trabalhos futuros.
\end{itemize}
