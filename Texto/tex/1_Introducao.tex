\section{Contextualização}
\qquad
Com o avanço de inúmeras tecnologias incluindo sensores, atuadores, computação em nuvem e o despontamento de
incontáveis dispositivos pequenos com capacidade de conexão com a Internet, muitos dos objetos presentes
na vida cotidiana tem obtido a capacidade de interoperabilidade com outros dispositivos.
\\ \null
\qquad
A \acrfull{IoT}, também chamada de Internet de Tudo ou até mesmo Internet Industrial, tem como objetivo primário
permitir que humanos e máquinas compreendam melhor o ambiente que os envolve, usando as informações geradas por
diversos dispositivos sensitivos. \acrshort{IoT} é um novo paradigma tecnológico planejado para ser uma rede
global de máquinas e dispositivos capazes de interagir entre si e com o ambiente ao seu redor.
A \acrshort{IoT} é reconhecida como uma das áreas mais importantes da tecnologia do futuro pelo fato de
poder ser implementada em diversos casos de uso.
\\ \null
\qquad Existem estimativas de que até 2020, 24 bilhões de dispositivos estarão interconectados, a partir desta quantidade
colossal de dispositivos, é esperado que a quantidade de dados seja superior a este número, em uma ordem de grandeza ainda maior.
Estes dados, se utilizados de forma correta, considerando contextos específicos e o nível de qualidade dessas informações, poderão
contribuir muito com avanços em pesquisas e serviços.
\section{Objetivo}
\qquad O objetivo geral do presente trabalho é o desenvolvimento de um sistema colaborativo de forma a gerenciar e avaliar a confiabilidade de dados
gerados em um ambiente de \acrlong{IoT}. O sistema colaborará com a avaliação dos dados obtidos, bem como a avaliação
dos sensores presentes na rede em que o sistema estiver implementado.
\subsection{Objetivos Específicos}
Para a realização do objetivo geral, os seguintes objetivos específicos devem ser atingidos:
\begin{itemize}
  \item Criar uma plataforma para criação de tarefas associadas à um tipo de sensor ou tipo de dado;
  \item Desenvolver um módulo de distribuição de tarefas, que atribuirá as tarefas aos colaboradores;
  \item Criar um módulo que realiza a pontuação para confiabilidade de um determinado sensor;
  %\item Desenvolver um ambiente \acrshort{IoT} em escala para teste de conceito.
\end{itemize}
\section{Estrutura do trabalho}
O presente trabalho é composto pelos seguintes capítulos:
\begin{itemize}
  \item \textbf{Capítulo 2 - Fundamentação Teórica:} apresentação dos conceitos necessários para a compreensão do
  trabalho. Neste capítulo serão abordados os seguintes temas:
  \begin{itemize}
    \item Sistemas colaborativos;
    \item \acrlong{IoT};
    \item Dados.
  \end{itemize}
  \item \textbf{Capítulo 3 - Metodologia:} detalhamento dos métodos utilizados para a realização do trabalho,
  apresentação de um sistema similar que serviu como inspiração para o presente trabalho.
  Explanação do processo de levantamento de requisitos, definição do sistema proposto e de sua arquitetura
  bem como a descrição do processo de desenvolvimento.
  \item \textbf{Capítulo 4 - Prova de Conceito:} exposição dos testes planejados e realizados no sistema e seus resultados, bem como a indicação de possíveis problemas e melhorias a serem realizadas.
  \item \textbf{Capítulo 5 - Conclusão:} apresentação das conclusões tomadas a partir dos resultados obtidos e
  sugestões para trabalhos futuros.
\end{itemize}
