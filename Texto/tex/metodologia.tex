\newcommand{\texCommand}[1]{\texttt{\textbackslash{#1}}}%

\newcommand{\exemplo}[1]{%
\vspace{\baselineskip}%
\noindent\fbox{\begin{minipage}{\textwidth}#1\end{minipage}}%
\\\vspace{\baselineskip}}%

\newcommand{\exemploVerbatim}[1]{%
\vspace{\baselineskip}%
\noindent\fbox{\begin{minipage}{\textwidth}%
#1\end{minipage}}%
\\\vspace{\baselineskip}}%


\quad Este capítulo descreve os métodos utilizados para a realização deste trabalho.

%%%%%%%%%%%%%%%%%%%%%%%%%%%%%%%%%%%%%%%%%%%%%%%%%%%%%%%%%%%%%%%%%%%%%%%%%%%%%%%%
%%%%%%%%%%%%%%%%%%%%%%%%%%%%%%%%%%%%%%%%%%%%%%%%%%%%%%%%%%%%%%%%%%%%%%%%%%%%%%%%
%%%%%%%%%%%%%%%%%%%%%%%%%%%%%%%%%%%%%%%%%%%%%%%%%%%%%%%%%%%%%%%%%%%%%%%%%%%%%%%%
\section{Revisão Sistemática}

\quad Foi definido um processo de revisão que consiste em três fases:
\begin{itemize}
  \item Planejar;
  \item Executar;
  \item Documentar.
\end{itemize}
\quad A fase de planejamento constitui-se na especificação do protocolo.
A fase de execução representa a coleta dos dados de forma a atender as especificações
exigidas na fase de planejamento.
A fase de documentação implica na consolidação dos dados obtidos.

\subsection{Planejamento}
\quad O objetivo desta revisão sistemática é a identificação de trabalhos acadêmicos
que expõem resultados, projeções, explicações ou elucidações sobre o tema de armazenamento
de metadados para \acrlong{IoT} com o propósito de que haja uma metodologia durante a
construção das referêcias do presente trabalho.

\subsubsection{Questões de Estudo}
\quad Esta revisão tem como objetivo responder às seuints questões:
\begin{itemize}
  \item Há estudos sobre o armazenamento de metadados para \acrlong{IoT}?
  \item Quais são os métodos mais utilizados para o armazenamento de metadados para \acrshort{IoT}?
  \item Quais são as características dos middlewares utilizados para estes estudos?
  % \item Qual a eficácia dos métodos expostos em termos de disponibilidade, confiabilidade?%duvida
  \item Quais aplicações os metadados estão sendo utilizados nos trabalhos
\end{itemize}

\subsubsection{Estratégia de Busca}
\quad É necessário determinar uma estratégia para a realização das buscas nas bases de dados escolhidas.

\paragraph{Definição da String de Busca}
\begin{itemize}
  \item \textbf{População}: A população é metadados para \acrshort{IoT}. Para procurar, foram utilizadas as palavas-chave 'Internet of things metadata' e 'IoT metadata';
  \item \textbf{Intervenção}: A intervenção é verificar middlewares e esquemas. Os termos utilizados para pesquisa foram 'middleware', 'management', 'model' e 'schema';
  \item \textbf{Comparação}: O foco deste trabalho não se limitou a estudos comparativos;
  \item \textbf{Resultado}: Tem-se como objetivo a procura de avaliações, definições, validações e implementações de middlewares e/ou esquemas utilizados em pesquisas científicas. Desta forma, obtemos as seguintes palavras-chave
  'validation', 'evaluation' e 'implementation'.
\end{itemize}

\quad A string de busca gerada é a seguinte:
(('Internet of things metadata') \textbf{OR} ('IoT metadata')) \textbf{AND} (('middleware') \textbf{OR} ('schema') \textbf{OR} ('management') \textbf{OR} ('model')) \textbf{AND} (('validation') \textbf{OR} ('evaluation') \textbf{OR} ('implementation'))

\paragraph{Fontes de Busca}
\subparagraph{}
 Foram escolhidas as seguintes bases digitais para que as buscas sejam realizadas:
\begin{itemize}
  \item Google Acadêmico (https://scholar.google.com.br/)
  \item JSTOR (https://www.jstor.org/)
  \item Periódicos CAPES (https://www.periodicos.capes.gov.br/)
\end{itemize}
\quad Bases escolhidas devido a sua relevância e sua grande abrangência sobre diversos temas.

\paragraph{Idioma}
\subparagraph{}
\quad Para este trabalho, o idioma de preferência para seleção de artigos será a língua inglesa, entretanto, trabalhos científicos escritos em
língua portuguesa não serão descartados, desde que atinjam os requisitos para inclusão.

\subsubsection{Seleção dos Estudos}
\paragraph{Critérios de Inclusão e Exclusão}
\subparagraph{Critérios de Inclusão}
\begin{itemize}
  \item Os trabalhos devem estar disponíveis nas bases de dados escolhidas previamente;
  \item Serão considerados apenas publicações posteriores à 2006 (ano o qual esta área começou a ser pesquisada de forma mais intensa), salvo em casos de fontes relevantes que contenham definições necessárias para a realização deste trabalho;
  \item O trabalho deve possuir menção a metadados em ambientes \acrlong{IoT}.
\end{itemize}

\subparagraph{Critérios de Exclusão}
\begin{itemize}
  \item Trabalhos publicados anteriormente à 2006, exceto fontes relevantes que contenham definições pertinentes para este estudo;
  \item Trabalhos não disponíveis nas bases de dados digitais escolhidas;
  \item O trabalho não possuir menção a metadados em ambientes \acrlong{IoT};
  \item Publicações que forem interpretadas como fora do escopo, ou seja, fogem do tema metadados em ambientes \acrshort{IoT};
  \item Trabalhos que não propõe, comparam ou avaliam métodos para o gerênciamento de metadados;
  \item Trabalhos duplicados, ou seja, publicados em mútiplas bases de dados.
\end{itemize}

\subsubsection{Processo de Seleção dos Estudos}
\quad Os artigos obtidos por meio da estratégia descrita acima passarão por um processo de avaliação sistêmico, com base nos critérios especificados anteriormente.
Desta forma, os artigos que atingirem os parâmetros serão adicionados à base de estudos da revisão sistemática. A estratégia para a pesquisa e seleção é:
\begin{enumerate}
  \item Pesquisa de trabalhos científicos nas bases de dados definidas utilizando as strings de busca;
  \item Leitura do título, resumo, palavras chave e data de publicação, aplicando os critérios de inclusão e exclusão definidos;
  \item Leitura da introdução e conclusão dos trabalhos que foram mantido na fase anterior;
  \item Os trabalhos selecionados na fase anterior serão lidos por completo, e as informações pertinentes serão coletadas destes trabalhos.
\end{enumerate}





%%%%%%%%%%%%%%%%%%%%%%%%%%%%%%%%%%%%%%%%%%%%%%%%%%%%%%%%%%%%%%%%%%%%%%%%%%%%%%%%
%%%%%%%%%%%%%%%%%%%%%%%%%%%%%%%%%%%%%%%%%%%%%%%%%%%%%%%%%%%%%%%%%%%%%%%%%%%%%%%%
%%%%%%%%%%%%%%%%%%%%%%%%%%%%%%%%%%%%%%%%%%%%%%%%%%%%%%%%%%%%%%%%%%%%%%%%%%%%%%%%
