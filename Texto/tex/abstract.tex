\emph{
Internet of Things (IoT) can be defined as the interconection of sensitive devices and actuators with a common purpose.
As the user of a system is included as an active participant of the IoT environment, it can contribute with the system with informations that the devices, by themselves, don't have the capacity to obtain such as variable environmental features, their specificities and the human perceptions about them.
 By using the informations provided by the human user it is possible that the system could gain more data about the environment, community and the space that it works in, improving it's capabilities of decision making after processing the data supplied by the network of sensors and the data provided by the users collaboratively.
  In the persuit of this objective, the following steps were achieved with the development of: (1) a small scale IoT environment;
  (2) a web system for aquisition, storage and access of informations provided by the IoT network;
  (3) a module that evaluates and creates a ranking of sensors using the informations provided by the sensors and the users.
}
