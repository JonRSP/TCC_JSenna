\emph{
Esse estudo trata do domínio da Internet das Coisas (IoT), que consiste na interconexão de dispositivos
 sensitivos e atuadores com a finalidade de atingir uma grande variedade de objetivos, por exemplo, automação predial e controle de processos produtivos.
 Ao incluir o usuário como participante ativo do ambiente de Internet das Coisas,
 permite-se a sua contribuição específica com os sistemas por meio da inserção de informações que os
  dispositivos, autonomamente, não têm a capacidade de coletar,
 como certas características ambientais variáveis ou abrangentes, suas especificidades ou até
  mesmo as percepções humanas sobre elas.
  Utilizando essas noções fornecidas pelo usuário humano, é possível que o sistema obtenha mais
  dados sobre o ambiente, sobre a comunidade e a espacialidade na qual está inserido, melhorando
  a sua capacidade em decidir assertivamente a partir do processamento conjunto entre os dados autonomamente
  coletados e as informações colaborativas fornecidas.
  Para tanto, nesse estudo, as seguintes fases foram realizadas com o desenvolvimento de: (1) um ambiente IoT em escala reduzida;
  (2) um sistema web para aquisição, armazenamento e acesso das informações;
  (3) um módulo que realize a avaliação e classificação dos sensores e dados,
  a partir das informações fornecidas pelos colaboradores.
}
