\emph{
Este estudo trata do domínio da Internet das Coisas, do inglês \textit{Internet of Things} (IoT), que consiste na interconexão de dispositivos
 sensitivos e atuadores com a finalidade de atingir uma grande variedade de objetivos, por exemplo, automação predial, controle de processos produtivos e transporte inteligente. % a vision, architectural...
 Ao incluir o usuário como participante ativo do ambiente de Internet das Coisas,
 permite-se a sua contribuição específica com os sistemas por meio da inserção de informações que os
  dispositivos, autonomamente, não têm a capacidade de coletar,
 como certas características ambientais variáveis ou abrangentes, suas especificidades ou até
  mesmo as percepções humanas sobre elas.
  Utilizando essas noções fornecidas pelo usuário humano, é possível que o sistema obtenha mais
  dados sobre o ambiente, sobre a comunidade e a espacialidade na qual está inserido, melhorando
  a sua capacidade em decidir assertivamente a partir do processamento conjunto entre os dados autonomamente
  coletados e as informações colaborativas fornecidas.
  Para tanto, nesse estudo, as seguintes fases foram realizadas com o desenvolvimento de: (1) um ambiente IoT em escala reduzida;
  (2) o sistema SenseHera, um sistema web para aquisição, armazenamento e acesso das informações;
  (3) um módulo que realize a avaliação da qualidade em termos de sua disponibilidade e regularidade dos sensores, a partir das informações coletadas pelos dispositivos e fornecidas pelos colaboradores.
}
