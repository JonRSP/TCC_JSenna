\section{Entidades Principais}
Lista de entidades principais implementadas no modelo do sistema SenseHera.
\subsection{Sensor}
A entidade "Sensor" armazena as informações sobre um dispositivo sensitivo, como descrição e pontuação.
\begin{center}
\captionof{table}{Tabela de dados da entidade Sensor.}\label{tabelaDSensor}
\begin{table}[h!]
\resizebox{\textwidth}{!}{%
\begin{tabular}{lrr}
\hline
Nome do Campo & \multicolumn{1}{l}{Tipo de dado} & \multicolumn{1}{l}{Descrição}                              \\ \hline
ID            & inteiro                          & Identificador do objeto; gerado automaticamente            \\ \hline
Descrição     & texto                            & Descrição atribuída ao sensor                              \\ \hline
Pontuação     & numérico                         & Pontuação do sensor; valor de 0 a 5 calculado pelo sistema %\\ \hline
\end{tabular}%
}
\end{table}
\end{center}


\subsection{Leitura}
A entidade "Leitura" armazena as leituras realizadas por um sensor, armazenando os valores coletados e associando-os com o sensor que os coletou e ao tipo de sensor referente a esta leitura.
\pagebreak
\begin{center}
\captionof{table}{Tabela de dados da entidade Leitura.}\label{tabelaDLeitura}
\begin{table}[h!]
\resizebox{\textwidth}{!}{%
\begin{tabular}{lrr}
\hline
Nome do Campo & \multicolumn{1}{l}{Tipo de dado} & \multicolumn{1}{l}{Descrição}                              \\ \hline
ID            & inteiro                          & Identificador do objeto; gerado automaticamente            \\ \hline
Momento       & data-hora                            & Instante em que a leitura foi realizada; gerada automaticamente                              \\ \hline
Valor         & numérico                         & Valor de uma leitura realizada pelo sensor \\ \hline
Sensor\_ID     & inteiro                         & \begin{tabular}[c]{@{}r@{}}Chave estrangeira;\\ identificador do sensor que realizou a leitura\end{tabular} \\ \hline
Tipo\_Sensor\_ID     & inteiro                         & \begin{tabular}[c]{@{}r@{}}Chave estrangeira;\\ identificador do tipo de sensor relacionado à leitura\end{tabular} \\ %\hline
\end{tabular}%
}
\end{table}
\end{center}
\subsection{Tipo\_Sensor}
A entidade "Tipo\_Sensor" foi criada para categorizar os sensores e armazenar informações sobre o tipo de sensor, como valores máximo e mínimo e unidade de leitura.
\begin{center}
\captionof{table}{Tabela de dados da entidade Tipo\_Sensor.}\label{tabelaDTipoSensor}
\begin{table}[h!]
\resizebox{\textwidth}{!}{%
\begin{tabular}{lrr}
\hline
Nome do Campo & \multicolumn{1}{l}{Tipo de dado} & \multicolumn{1}{l}{Descrição}                              \\ \hline
ID            & inteiro                          & Identificador do objeto; gerado automaticamente            \\ \hline
Descrição     & texto                            & Descrição atribuída ao tipo de sensor                              \\ \hline
Unidade     & texto                         & Unidade de leitura \\ \hline
Minimo     & numérico                         & Valor mínimo esperado para um tipo de sensor \\ \hline
Maximo     & numérico                         & Valor máximo esperado para um tipo de sensor \\ %\hline
\end{tabular}%
}
\end{table}
\end{center}
\subsection{Pergunta}
A entidade "Pergunta" foi criada paraarmazenar as perguntas a serem respondidas pelos usuários colaboradores.
\begin{center}
\captionof{table}{Tabela de dados da entidade Pergunta.}\label{tabelaDPergunta}
\begin{table}[h!]
\resizebox{\textwidth}{!}{%
\begin{tabular}{lrr}
\hline
Nome do Campo & \multicolumn{1}{l}{Tipo de dado} & \multicolumn{1}{l}{Descrição}                              \\ \hline
ID            & inteiro                          & Identificador do objeto; gerado automaticamente            \\ \hline
Descrição     & texto                            & Texto da pergunta                              \\ \hline
Tipo\_Pergunta\_ID     & inteiro                         & \begin{tabular}[c]{@{}r@{}}Chave estrangeira;\\ identificador do tipo de pergunta relacionado à pergunta\end{tabular} \\ %\hline
\end{tabular}%
}
\end{table}
\end{center}
\subsection{Tipo\_Pergunta}
A entidade "Tipo\_Pergunta"\ implementa o que foi proposto na Figura \ref{funcionamento2}, dividindo as perguntas sobre as características do ambiente das perguntas sobre as sensações do usuário colaborador. As\ perguntas podem ser utilizadas para se obter o contexto espacial ou psicológico.
\begin{center}
\captionof{table}{Tabela de dados da entidade Tipo\_Pergunta.}\label{tabelaDTipoPergunta}
\begin{table}[h!]
\resizebox{\textwidth}{!}{%
\begin{tabular}{lrr}
\hline
Nome do Campo & \multicolumn{1}{l}{Tipo de dado} & \multicolumn{1}{l}{Descrição}                              \\ \hline
ID            & inteiro                          & Identificador do objeto; gerado automaticamente            \\ \hline
Descrição     & texto                            & Descrição atribuída ao tipo de pergunta                              %\\ \hline
\end{tabular}%
}
\end{table}
\end{center}
\subsection{Resposta\_Possível}
A entidade "Resposta\_Possível"\ foi criada com o intuito de evitar problemas durante a fase de colaboração humana com o sistema. Ao fixar as possibilidades de respostas para cada pergunta, evitam-se problemas como, por exemplo, erros de ortografia e erros na compreensão da pergunta.
\begin{center}
\captionof{table}{Tabela de dados da entidade Resposta\_Possível.}\label{tabelaDRespostaPossivel}
\begin{table}[h!]
\resizebox{\textwidth}{!}{%
\begin{tabular}{lrr}
\hline
Nome do Campo & \multicolumn{1}{l}{Tipo de dado} & \multicolumn{1}{l}{Descrição}                              \\ \hline
ID            & inteiro                          & Identificador do objeto; gerado automaticamente            \\ \hline
Descrição     & texto                            & Texto atribuído à resposta possível                              %\\ \hline
\end{tabular}%
}
\end{table}
\end{center}
\subsection{Resposta}
A entidade "Resposta" foi criada para agrupar a informação fornecidas pelo usuário, a pergunta que foi respondida, a qual sensor essa resposta é associada e o instante em que a pergunta foi respondida.
\pagebreak
\begin{center}
\captionof{table}{Tabela de dados da entidade Resposta.}\label{tabelaDResposta}
\begin{table}[h!]
\resizebox{\textwidth}{!}{%
\begin{tabular}{lrr}
\hline
Nome do Campo & \multicolumn{1}{l}{Tipo de dado} & \multicolumn{1}{l}{Descrição}                              \\ \hline
ID            & inteiro                          & Identificador do objeto; gerado automaticamente            \\ \hline
Momento       & data-hora                            & Descrição atribuída ao sensor                              \\ \hline
Sensor\_ID     & inteiro                         & \begin{tabular}[c]{@{}r@{}}Chave estrangeira;\\ identificador do sensor relacionado à resposta\end{tabular} \\ \hline
Pergunta\_ID     & inteiro                         & \begin{tabular}[c]{@{}r@{}}Chave estrangeira;\\ identificador da pergunta relacionada à resposta\end{tabular} \\ \hline
Resposta\_Possivel\_ID     & inteiro                         & \begin{tabular}[c]{@{}r@{}}Chave estrangeira;\\ identificador da resposta possivel relacionada à resposta\end{tabular} %\\ \hline
\end{tabular}%
}
\end{table}
\end{center}

\section{Tabelas Auxiliares}
Tabelas auxiliares necessárias para o funcionamento do sistema SenseHera.
\subsection{TB\_Sensor\_Tipo\_Sensor}
Esta tabela é criada devido ao relacionamento n para n entre as entidades Sensor e Tipo\_Sensor, em que um sensor pode ter múltiplos tipos de sensor associados e um tipo de sensor pode ter múltiplos dispositivos associados.
\begin{center}
\captionof{table}{Tabela de dados da tabela TB\_Sensor\_Tipo\_Sensor.}\label{tabelaDTBSensorTipoSensor}
\begin{table}[h!]
\resizebox{\textwidth}{!}{%
\begin{tabular}{lrr}
\hline
Nome do Campo & \multicolumn{1}{l}{Tipo de dado} & \multicolumn{1}{l}{Descrição}                              \\ \hline
Sensor\_ID     & inteiro           & \begin{tabular}[c]{@{}r@{}}Chave estrangeira e primária;\\ identificador do sensor relacionado ao tipo de sensor\end{tabular}                              \\ \hline
Tipo\_Sensor\_ID     & inteiro                         & \begin{tabular}[c]{@{}r@{}}Chave estrangeira e primária;\\ identificador do tipo de sensor relacionado ao sensor\end{tabular} %\\ \hline
\end{tabular}%
}
\end{table}
\end{center}
\subsection{TB\_Pergunta\_Tipo\_Sensor}
Esta tabela é criada devido ao relacionamento n para n entre as entidades Pergunta e Tipo\_Sensor, em que uma pergunta pode ter múltiplos tipos de sensor associados e um tipo de sensor pode ter múltiplas questões associadas.
\pagebreak
\begin{center}
\captionof{table}{Tabela de dados da tabela TB\_Pergunta\_Tipo\_Sensor.}\label{tabelaDTBPerguntaTipoSensor}
\begin{table}[h!]
\resizebox{\textwidth}{!}{%
\begin{tabular}{lrr}
\hline
Nome do Campo & \multicolumn{1}{l}{Tipo de dado} & \multicolumn{1}{l}{Descrição}                              \\ \hline
Tipo\_Sensor\_ID     & inteiro                         & \begin{tabular}[c]{@{}r@{}}Chave estrangeira e primária;\\ identificador do tipo de sensor relacionado à pergunta\end{tabular} \\ \hline
Pergunta\_ID     & inteiro           & \begin{tabular}[c]{@{}r@{}}Chave estrangeira e primária;\\ identificador da pergunta relacionada ao tipo de sensor\end{tabular}                              %\\ \hline
\end{tabular}%
}
\end{table}
\end{center}
\subsection{TB\_Pergunta\_Resposta\_Possivel}
Esta tabela é criada devido ao relacionamento n para n entre as entidades Pergunta e Resposta\_Possivel, em que uma pergunta pode ter múltiplas respostas possíveis associadas e uma resposta possível pode ter múltiplas questões associadas.
\begin{center}
\captionof{table}{Tabela de dados da tabela TB\_Pergunta\_Resposta\_Possivel.}\label{tabelaDTBPerguntaRespostaPossivel}
\begin{table}[h!]
\resizebox{\textwidth}{!}{%
\begin{tabular}{lrr}
\hline
Nome do Campo & \multicolumn{1}{l}{Tipo de dado} & \multicolumn{1}{l}{Descrição}                              \\ \hline
Pergunta\_ID     & inteiro           & \begin{tabular}[c]{@{}r@{}}Chave estrangeira e primária;\\ identificador da pergunta relacionada à resposta possível\end{tabular}                              \\ \hline
Resposta\_Possivel\_ID     & inteiro                         & \begin{tabular}[c]{@{}r@{}}Chave estrangeira e primária; \\identificador da resposta possível relacionada à pergunta\end{tabular} %\\ \hline
\end{tabular}%
}
\end{table}
\end{center}
