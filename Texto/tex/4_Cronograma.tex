% Neste capítulo será descrito o cronograma de atividades a serem realizadas.
%
% \section{Cronograma Estudos Em (2/2018)}
% \begin{itemize}
%   \item Construção do ambiente \acrshort{IoT} em escala; (20/7 - 1/8)
%   \item Definição dos requisitos para o sistema; (1/8 - 5/8)
%   \item Construção do sistema; (5/8 - 1/12)
%   \item Testes; (5/8 - 1/12)
%   \item Aquisição dos resultados. (1/12 - 25/1)
% \end{itemize}
\newpage
 \section{Cronograma Projeto Final em Engenharia de Computação 2 (1/2019)}

\begin{itemize}
  \item Escrever sobre o sistema construído; (1/3 - 1/4)
  \item Escrever a conclusão sobre o trabalho; (1/4 - 20/4)
  \item Escrever resumo e abstract; (20/4 - 30/4)
  \item Produzir a apresentação de slides; (30/4 - 10/5)
  \item Correções; (10/5 - ?)
  \item Apresentar. (?)
\end{itemize}

\newpage
 \section{Definição do Problema}
 Em ambientes de \acrlong{IoT} é necessário que a qualidade dos dados fornecidos pelos sensores seja a mais alta possível para que, a partir desses dados,
 decisões sejam tomadas e ações sejam realizadas.  Não há, entretanto, no ambiente acadêmico, estudos sobre como enfrentar o problema de forma colaborativa (usuário-sistema).
 A falta de estudos sobre o tema impedem a suposição de que uma abordagem colaborativa para a resolução do problema seja funcional ou não.



 \newpage
  \section{Solução Proposta}
  O problema sobre a qualidade dos dados fornecidos pelos sensores será resolvido de forma colaborativa; o usuário contribuirá com informações sobre o ambiente e sensações. Estas informações, por sua vez, servirão para validar os dados e fornecer ao sistema metadados sobre as medições que estão sendo feitas pelo ambiente \acrshort{IoT}.
 Ao resolver este problema de forma colaborativa, o usuário final atua no sistema, o que leva a uma participação do personagem mais interessado no funcionamento da rede \acrlong{IoT}, possivelmente gerando resultados positivos.
