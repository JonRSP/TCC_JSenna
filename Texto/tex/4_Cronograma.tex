% Neste capítulo será descrito o cronograma de atividades a serem realizadas.
%
% \section{Cronograma Estudos Em (2/2018)}
% \begin{itemize}
%   \item Construção do ambiente \acrshort{IoT} em escala; (20/7 - 1/8)
%   \item Definição dos requisitos para o sistema; (1/8 - 5/8)
%   \item Construção do sistema; (5/8 - 1/12)
%   \item Testes; (5/8 - 1/12)
%   \item Aquisição dos resultados. (1/12 - 25/1)
% \end{itemize}
\newpage
 \section{Cronograma Projeto Final em Engenharia de Computação 2 (1/2019)}

\begin{itemize}
  \item Escrever sobre o sistema construído; (1/3 - 31/5)
  \item Escrever a conclusão sobre o trabalho; (1/6 - 15/6)
  %\item Escrever resumo e abstract; (1/6 - 15/6)
  \item Produzir a apresentação de slides; (15/6 - 21/6)
\end{itemize}

\newpage
 \section{Definição do Problema}
 Em uma rede \acrshort{IoT} há um enorme volume de dados gerados por uma grande quantidade de sensores. Para que sistemas automatizados ou usuários não se equivoquem
  na tomada de decisões por causa de falhas em leituras ou erros na transmissão de dados,
  é necessário que as informações coletadas por estes sensores passem por uma avaliação de forma a
 classificar os sensores pela sua qualidade e consistência.
 A falta da noção de qualificação de um sensor em termos de sua precisão e regularidade prejudica as decisões tomadas pelo sistema,
  não garantindo que estas sejam as mais acertivas possíveis.
Este projeto visa o desenvolvimento de um sistema que possibilite a avaliação e classificação da qualidade de sensores de forma colaborativa.


 \newpage
  \section{Solução Proposta}
  Desenvolver um ambiente \acrshort{IoT} em escala reduzida para teste de conceito;
  Desenvolver um sistema web para aquisição, armazenamento e acesso das informações geradas pelos sensores;
  Desenvolver um módulo que realize a avaliação de qualidade de sensores e dados, a partir dos dados fornecidos pelos colaboradores.
  As figuras \ref{funcionamento1} e \ref{funcionamento2} servem para ilustrar a solução.
  \figura[!h]{solucao1.jpg}{Esquema de funcionamento do sistema}{funcionamento1}{scale=0.9}
  \\
  \\\\
  Na figura \ref{funcionamento1} a interação I consiste no envio dos dados coletados pelos sensores para o servidor.
  A interação II resulta do acesso dos usuários às informações contidas no sistema e o envio de suas percepções
  e sensações sobre o ambiente para o servidor.
\newpage
  \figura[!h]{solucao2.jpg}{Esquema de participação dos usuários no sistema}{funcionamento2}{scale=0.7}



  A figura \ref{funcionamento2} indica a forma em que o usuário pode auxiliar com informações importantes
  sobre o suas percepções e sobre o ambiente, fornecendo uma contextualização para o sistema, como por exemplo:
  \begin{itemize}
    \item Quantidade de aberturas;
    \item Fontes de calor;
    \item Percepções de frio e calor.
  \end{itemize}
